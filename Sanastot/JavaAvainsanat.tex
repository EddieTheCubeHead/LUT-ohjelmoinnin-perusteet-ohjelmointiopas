\newglossaryentry{java:bool}
{
    type=java,
    name=bool,
    description={Primitiivinen tietotyyppi, joka sisältää totuusarvon (true tai false)},
    see={primitiivinen tietotyyppi}
}

\newglossaryentry{java:byte}
{
    type=java,
    name=byte,
    description={Primitiivinen tietotyyppi, joka sisältää tavun kokoisen merkillisen
kokonaisluvun},
    see={primitiivinen tietotyyppi}
}

\newglossaryentry{java:short}
{
    type=java,
    name=short,
    description={Primitiivinen tietotyyppi, joka sisältää kahden tavun kokoisen merkillisen
kokonaisluvun},
    see={primitiivinen tietotyyppi}
}

\newglossaryentry{java:int}
{
    type=java,
    name=int,
    description={Primitiivinen tietotyyppi, joka sisältää 32 bitin kokoisen merkillisen
kokonaisluvun},
    see={primitiivinen tietotyyppi}
}

\newglossaryentry{java:long}
{
    type=java,
    name=long,
    description={Primitiivinen tietotyyppi, joka sisältää 64 bitin kokoisen merkillisen
kokonaisluvun},
    see={primitiivinen tietotyyppi}
}

\newglossaryentry{java:float}
{
    type=java,
    name=float,
    description={Primitiivinen tietotyyppi, joka sisältää 32 bitin kokoisen liukumaesitetyn
desimaaliluvun},
    see={primitiivinen tietotyyppi}
}

\newglossaryentry{java:double}
{
    type=java,
    name=double,
    description={Primitiivinen tietotyyppi, joka sisältää 64 bitin kokoisen liukumaesitetyn
desimaaliluvun},
    see={primitiivinen tietotyyppi}
}

\newglossaryentry{java:char}
{
    type=java,
    name=char,
    description={Primitiivinen tietotyyppi, joka sisältää kahden tavun kokoisen unicode-koodatun
merkin esitettynä merkittömänä kokonaislukuna},
    see={primitiivinen tietotyyppi}
}

\newglossaryentry{java:String}
{
    type=java,
    name=String,
    description={Javan standardikirjaston merkkijonoimplementaatioluokka. Suositellaan
käytettäväksi merkkijonojen säilömiseen koodissa. Pystyy säilömään dynaamisen merkkijonon, jonka
alustus- tai maksimikokoa ei tarvitse määrittää erikseen. Lisäksi sisältää lukuisia merkkijonon
käsittelyä ja muokkausta helpottavia metodeja}
}

\newglossaryentry{java:private}
{
	type=java,
	name=private,
	description={Näkyvyysmääre, joka määrittää ominaisuuden olevan käytettävissä vain ominaisuuden
omistaman luokan sisällä}
}

\newglossaryentry{java:protected}
{
	type=java,
	name=protected,
	description={Näkyvyysmääre, joka määrittää ominaisuuden olevan käytettävissä ominaisuuden
omistavan luokan sisältävässä packagessa ja kaikissa ominaisuuden omistavan luokan perivissä
luokissa}
}

\newglossaryentry{java:public}
{
	type=java,
	name=public,
	description={Näkyvyysmääre, joka määrittää ominaisuuden olevan käytettävissä kaikkialla
ohjelmistossa}
}

\newglossaryentry{java:static}
{
	type=java,
	name=static,
	description={Staattinen metodi tai muuttuja näkyy kaikille sen omistavan luokan instansseille
jaetusti. Tunnetaan luokkamuuttujana tai luokkametodina}
}

\newglossaryentry{java:void}
{
	type=java,
	name=void,
	description={Muuttujan paluuarvotyyppi muuttujalle, joka ei palauta mitään}
}

\newglossaryentry{java:main}
{
	type=java,
	name=main,
	description={Varattu metodinimi metodille, jonka Java ajaa ensimmäisenä ajaessaan ohjelmistoa.
Ohjelmistossa voi olla useampi-main niminen metodi, mutta vain määritellyn juuriluokan main-metodi
ajetaan. Metodin täytyy olla muotoa public static void ja ottaa yksi taulukko String-luokan
instansseja},
	see={metodi,taulukko}
}

\newglossaryentry{java:class}
{
	type=java,
	name=class,
	description={Avainsana, joka aloittaa luokan määritelmän.}
}

\newglossaryentry{java:if}
{
	type=java,
	name=if,
	description={Avainsana, jota seuraa normaaleihin sulkuihin suljettu totuusarvo ja
kaarisulkuihin suljettu koodin osa. Tämä koodin osa ajetaan vain jos annettu totuusarvo on
tosi. Voidaan yhdistää else-lauseeseen},
	see=java:else
}

\newglossaryentry{java:else}
{
	type=java,
	name=else,
	description={Avainsana, joka voi seurata if-lauseella määriteltyä ehdollista koodin osaa.
Else-lausetta seuraa kaarisulkuihin suljettu koodin osa. Tämä koodin osa ajetaan vain, jos
else-lausetta edeltänyttä ehdollista koodin osaa ei ajettu},
	see=java:if
}

\newglossaryentry{java:new}
{
	type=java,
	name=new,
	description={Avainsana, joka luo uuden instanssin luokasta. Avainsanaa seuraa aina toivotun
luokan nimi, jonka perässä on sulkuihin suljettuna toivotut rakentajalle annettavat parametrit,
metodikutsun tapaan},
	see=rakentaja
}

\newglossaryentry{java:Scanner}
{
	type=java,
	name=Scanner,
	description={Javan standardikirjaston luokka halutun lähteen lukemiseen ja parsimiseen.
Rakentaja ottaa parsittavan stream-olion. Sisältää metodeja eri tietotyyppien parsimiseen
annetusta stream-oliosta}
}

\newglossaryentry{java:BufferedReader}
{
	type=java,
	name=BufferedReader,
	description={Javan standardikirjaston luokka puskuroitua merkkijonostream-olion lukemista
varten. Rakentaja ottaa luettavan stream-olion ja vapaaehtoisena argumenttina puskurin koon.
Käytetään isompien syötteiden lukemiseen, koska syötteen kääriminen BufferedReader-olioon
luettaessa vähentää turhia lukuoperaatioita. Scanner-luokka on suositellumpi esimerkiksi
lyhyen käyttäjäsyötteen lukemiseen},
	see=java:Scanner
}

\newglossaryentry{java:InputStreamReader}
{
	type=java,
	name=InputStreamReader,
	description={Javan standardikirjaston luokka tavujonon merkkijonoksi muuttamista varten.
Rakentaja ottaa muutettavan stream-olion ja vapaaehtoisesti merkkisetin}
}

\newglossaryentry{java:System}
{
	type=java,
	name=System,
	description={Javan standardikirjaston luokka, joka sisältää luokkametodeja ja luokkamuuttuja
systeemirajapintojen, kuten ympäristömuuttujien, tulosteen ja syötteen käyttöön},
	see={luokkamuuttuja,luokkametodi}
}

\newglossaryentry{java:ArrayList}
{
	type=java,
	name=ArrayList,
	description={Javan standardikirjaston luokka, joka toteuttaa List-rajapinnan. Geneerinen
säilötyn luokan suhteen, eli voi säilöä minkä tahansa luokan olioita, kunhan säilötyt oliot ovat
kaikki saman luokan instansseja. Ei voi säilöä primitiivisiä tietotyyppejä. Luodaan
new-avainsanalla  normaalin geneerisen luokan tapaan. Merkkijonoja säilövä ArrayList-instanssi
voidaan siis luoda seuraavalla koodinpätkällä:\newline{}"new ArrayList<String>();"},
	see={java:List,geneerinen luokka,java:new}
}

\newglossaryentry{java:List}
{
	type=java,
	name=List,
	description={Javan standardikirjaston rajapinta, jonka instanssit ovat järjestettyjä
kokoelmia. Kaikki List-rajapinnan toteuttavat luokat tukevat iterointia, indeksipohjaista hakua
ja lisäystä ja vertailua. Yleisimmin käytetty List-rajapinnan toteuttava luokka on ArrayList,
mutta esimerkiksi myös LinkedList ja ArrayQueue toteuttavat rajapinnan},
	see={java:ArrayList,java:interface,rajapinta}
}

\newglossaryentry{java:interface}
{
	type=java,
	name=interface,
	description={Avainsana, joka aloittaa rajapintaluokan määritelmän Javassa normaalin class-
avainsanan sijaan. Rajapintaluokka määrittää signatuurit metodeille, jotka kaikkien rajapinnan
implementoivien luokkien on toteutettava. Näin luotu rajapintaluokka ei voi sisältää datakenttiä
eikä sitä voi käyttää olion muottina. Sen sijaan toinen luokka voi implementoida rajapinnan
jolloin implementoiva luokka lupaa muulle ohjelmistolle tarjoavansa rajapinnan määrittelemät
funktiot osana toiminnallisuuttaan},
	see={rajapinta,java:implements}
}

\newglossaryentry{java:implements}
{
	type=java,
	name=implements,
	description={Avainsana, joka määrittää luokan implementoivan jonkin rajapinnan. Tällöin
kyseiseen luokkaan on määriteltävä metodit, jotka vastaavat kyseisessä rajapinnassa määriteltyjä
metodeja signatuuriltaan. Esimerkiksi luokka "PasswordLogin", joka implementoi
"Authenticator"-rajapinnan määritellään seuraavasti:\newline{}
"class PasswordLogin implements Authenticator..."},
	see={java:class,java:interface}
}

\newglossaryentry{java:HashMap}
{
	type=java,
	name=HashMap,
	description={Javan standardikirjaston luokka, joka toteuttaa hajautuskartta-tietorakenteen.
Geneerinen luokka, joka ottaa erikseen avaimen ja arvon tietotyypin. Osaa säilöä myös null-arvoja
avaimeen tai arvoon. Luodaan new-avainsanalla normaalin geneerisen luokan tapaan. HashMap, jossa
on merkkijonoavaimet ja kokonaislukuarvot voidaan siis luoda seuraavalla koodinpätkällä:
\newline{}"new HashMap<String, int>"\newline{}Huomioi, että avaimen tyyppi annetaan ensimmäisenä},
	see={geneerinen luokka,hajautustaulu}
}