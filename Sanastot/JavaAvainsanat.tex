\newglossaryentry{java:bool}
{
    type=java,
    name=bool,
    description={Primitiivinen tietotyyppi, joka sisältää totuusarvon (true tai false)},
    see={primitiivinen tietotyyppi}
}

\newglossaryentry{java:byte}
{
    type=java,
    name=byte,
    description={Primitiivinen tietotyyppi, joka sisältää tavun kokoisen merkillisen
kokonaisluvun},
    see={primitiivinen tietotyyppi}
}

\newglossaryentry{java:short}
{
    type=java,
    name=short,
    description={Primitiivinen tietotyyppi, joka sisältää kahden tavun kokoisen merkillisen
kokonaisluvun},
    see={primitiivinen tietotyyppi}
}

\newglossaryentry{java:int}
{
    type=java,
    name=int,
    description={Primitiivinen tietotyyppi, joka sisältää 32 bitin kokoisen merkillisen
kokonaisluvun},
    see={primitiivinen tietotyyppi}
}

\newglossaryentry{java:long}
{
    type=java,
    name=long,
    description={Primitiivinen tietotyyppi, joka sisältää 64 bitin kokoisen merkillisen
kokonaisluvun},
    see={primitiivinen tietotyyppi}
}

\newglossaryentry{java:float}
{
    type=java,
    name=float,
    description={Primitiivinen tietotyyppi, joka sisältää 32 bitin kokoisen liukumaesitetyn
desimaaliluvun},
    see={primitiivinen tietotyyppi}
}

\newglossaryentry{java:double}
{
    type=java,
    name=double,
    description={Primitiivinen tietotyyppi, joka sisältää 64 bitin kokoisen liukumaesitetyn
desimaaliluvun},
    see={primitiivinen tietotyyppi}
}

\newglossaryentry{java:char}
{
    type=java,
    name=char,
    description={Primitiivinen tietotyyppi, joka sisältää kahden tavun kokoisen unicode-koodatun
merkin esitettynä merkittömänä kokonaislukuna},
    see={primitiivinen tietotyyppi}
}

\newglossaryentry{java:String}
{
    type=java,
    name=String,
    description={Javan standardikirjaston merkkijonoimplementaatioluokka. Suositellaan
käytettäväksi merkkijonojen säilömiseen koodissa. Pystyy säilömään dynaamisen merkkijonon, jonka
alustus- tai maksimikokoa ei tarvitse määrittää erikseen. Lisäksi sisältää lukuisia merkkijonon
käsittelyä ja muokkausta helpottavia metodeja}
}

\newglossaryentry{java:private}
{
	type=java,
	name=private,
	description={Näkyvyysmääre, joka määrittää ominaisuuden olevan käytettävissä vain ominaisuuden
omistaman luokan sisällä}
}

\newglossaryentry{java:protected}
{
	type=java,
	name=protected,
	description={Näkyvyysmääre, joka määrittää ominaisuuden olevan käytettävissä ominaisuuden
omistavan luokan sisältävässä packagessa ja kaikissa ominaisuuden omistavan luokan perivissä
luokissa}
}

\newglossaryentry{java:public}
{
	type=java,
	name=public,
	description={Näkyvyysmääre, joka määrittää ominaisuuden olevan käytettävissä kaikkialla
ohjelmistossa}
}

\newglossaryentry{java:static}
{
	type=java,
	name=static,
	description={Staattinen metodi tai muuttuja näkyy kaikille sen omistavan luokan instansseille
jaetusti. Tunnetaan luokkamuuttujana tai luokkametodina}
}

\newglossaryentry{java:void}
{
	type=java,
	name=void,
	description={Muuttujan paluuarvotyyppi muuttujalle, joka ei palauta mitään}
}

\newglossaryentry{java:main}
{
	type=java,
	name=main,
	description={Varattu metodinimi metodille, jonka Java ajaa ensimmäisenä ajaessaan ohjelmistoa.
Ohjelmistossa voi olla useampi-main niminen metodi, mutta vain määritellyn juuriluokan main-metodi
ajetaan. Metodin täytyy olla muotoa public static void ja ottaa yksi taulukko String-luokan
instansseja},
	see={metodi,taulukko}
}

\newglossaryentry{java:class}
{
	type=java,
	name=class,
	description={Avainsana, joka aloittaa luokan määritelmän}
}

\newglossaryentry{java:if}
{
	type=java,
	name=if,
	description={Avainsana, jota seuraa normaaleihin sulkuihin suljettu totuusarvo ja
kaarisulkuihin suljettu koodin osa. Tämä koodin osa ajetaan vain jos annettu totuusarvo on
tosi. Voidaan yhdistää else-lauseeseen},
	see=java:else
}

\newglossaryentry{java:else}
{
	type=java,
	name=else,
	description={Avainsana, joka voi seurata if-lauseella määriteltyä ehdollista koodin osaa.
Else-lausetta seuraa kaarisulkuihin suljettu koodin osa. Tämä koodin osa ajetaan vain, jos
else-lausetta edeltänyttä ehdollista koodin osaa ei ajettu},
	see=java:if
}

\newglossaryentry{java:new}
{
	type=java,
	name=new,
	description={Avainsana, joka luo uuden instanssin luokasta. Avainsanaa seuraa aina toivotun
luokan nimi, jonka perässä on sulkuihin suljettuna toivotut rakentajalle annettavat parametrit,
metodikutsun tapaan},
	see=rakentaja
}

\newglossaryentry{java:Scanner}
{
	type=java,
	name=Scanner,
	description={Javan standardikirjaston luokka halutun lähteen lukemiseen ja parsimiseen.
Rakentaja ottaa parsittavan stream-olion. Sisältää metodeja eri tietotyyppien parsimiseen
annetusta stream-oliosta}
}

\newglossaryentry{java:BufferedReader}
{
	type=java,
	name=BufferedReader,
	description={Javan standardikirjaston luokka puskuroitua merkkijonostream-olion lukemista
varten. Rakentaja ottaa luettavan stream-olion ja vapaaehtoisena argumenttina puskurin koon.
Käytetään isompien syötteiden lukemiseen, koska syötteen kääriminen BufferedReader-olioon
luettaessa vähentää turhia lukuoperaatioita. Scanner-luokka on suositellumpi esimerkiksi
lyhyen käyttäjäsyötteen lukemiseen},
	see=java:Scanner
}

\newglossaryentry{java:InputStreamReader}
{
	type=java,
	name=InputStreamReader,
	description={Javan standardikirjaston luokka tavujonon merkkijonoksi muuttamista varten.
Rakentaja ottaa muutettavan stream-olion ja vapaaehtoisesti merkkisetin}
}

\newglossaryentry{java:System}
{
	type=java,
	name=System,
	description={Javan standardikirjaston luokka, joka sisältää luokkametodeja ja luokkamuuttuja
systeemirajapintojen, kuten ympäristömuuttujien, tulosteen ja syötteen käyttöön},
	see={luokkamuuttuja,luokkametodi}
}

\newglossaryentry{java:ArrayList}
{
	type=java,
	name=ArrayList,
	description={Javan standardikirjaston luokka, joka toteuttaa List-rajapinnan. Geneerinen
säilötyn luokan suhteen, eli voi säilöä minkä tahansa luokan olioita, kunhan säilötyt oliot ovat
kaikki saman luokan instansseja. Ei voi säilöä primitiivisiä tietotyyppejä. Luodaan
new-avainsanalla  normaalin geneerisen luokan tapaan. Merkkijonoja säilövä ArrayList-instanssi
voidaan siis luoda seuraavalla koodinpätkällä:\newline{}"new ArrayList<String>();"},
	see={java:List,geneerinen luokka,java:new}
}

\newglossaryentry{java:List}
{
	type=java,
	name=List,
	description={Javan standardikirjaston rajapinta, jonka instanssit ovat järjestettyjä
kokoelmia. Kaikki List-rajapinnan toteuttavat luokat tukevat iteraattorin luontia, instanssien
vertailua ja indeksipohjaista hakua ja lisäystä. Yleisimmin käytetty List-rajapinnan toteuttava
luokka on ArrayList, mutta esimerkiksi myös LinkedList ja ArrayQueue toteuttavat rajapinnan},
	see={java:ArrayList,java:interface,rajapinta}
}

\newglossaryentry{java:interface}
{
	type=java,
	name=interface,
	description={Avainsana, joka aloittaa rajapintaluokan määritelmän Javassa normaalin class-
avainsanan sijaan. Rajapintaluokka määrittää signatuurit metodeille, jotka kaikkien rajapinnan
implementoivien luokkien on toteutettava. Näin luotu rajapintaluokka ei voi sisältää datakenttiä
eikä sitä voi käyttää olion muottina. Sen sijaan toinen luokka voi implementoida rajapinnan
jolloin implementoiva luokka lupaa muulle ohjelmistolle tarjoavansa rajapinnan määrittelemät
funktiot osana toiminnallisuuttaan},
	see={rajapinta,java:implements}
}

\newglossaryentry{java:implements}
{
	type=java,
	name=implements,
	description={Avainsana, joka määrittää luokan implementoivan jonkin rajapinnan. Tällöin
kyseiseen luokkaan on määriteltävä metodit, jotka vastaavat kyseisessä rajapinnassa määriteltyjä
metodeja signatuuriltaan. Esimerkiksi luokka "PasswordLogin", joka implementoi
"Authenticator"-rajapinnan määritellään seuraavasti:\newline{}
"class PasswordLogin implements Authenticator..."},
	see={java:class,java:interface}
}

\newglossaryentry{java:HashMap}
{
	type=java,
	name=HashMap,
	description={Javan standardikirjaston luokka, joka toteuttaa hajautuskartta-tietorakenteen.
Geneerinen luokka, joka ottaa erikseen avaimen ja arvon tietotyypin. Osaa säilöä myös null-arvoja
avaimeen tai arvoon. Luodaan new-avainsanalla normaalin geneerisen luokan tapaan. HashMap, jossa
on merkkijonoavaimet ja kokonaislukuarvot voidaan siis luoda seuraavalla koodinpätkällä:
\newline{}"new HashMap<String, int>"\newline{}Huomioi, että avaimen tyyppi annetaan ensimmäisenä},
	see={geneerinen luokka,hajautustaulu}
}

\newglossaryentry{java:for}
{
	type=java,
	name=for,
	description={Avainsana, joka alustaa for-loopin tai foreach-loopin. Avainsanaa seuraa loopin
toistoehto sulkuihin suljettuna. Toistoehto määrittää miten kauan looppia ajetaan. For-loopissa
ehto on kolmiosainen: ensimmäinen osa kertoo mikä muuttuja määrittää toistoehdon, toinen millä
kyseisen muuttujan arvoilla toistoa jatketaan ja kolmas mitä muuttujalle tehdään jokaisen toiston
jälkeen. Foreach-loopissa toistoehto määrittää kokoelman, jonka yli looppi iteroi. Kummassakin
looppityypissä toistoehdon jälkeen seuraa loopin runko suljettuna kaarisulkuihin. Molemmat loopit
tukevat continue- ja break-avainsanoja},
	see={java:continue,java:break}
}

\newglossaryentry{java:while}
{
	type=java,
	name=while,
	description={Avainsana, joka luo while-loopin. Avainsanaa seuraa loopin toistoehto sulkuihin
suljettuna totuusarvona. Tämän jälkeen annetaan loopin runko kaarisulkeisiin suljettuna. Runkoa
toistetaan niin kauan, kunnes annettu totuusarvo on epätosi. Tukee continue- ja 
break-avainsanoja. Määrittää toistoehdon do-loopissa},
	see={java:continue,java:break,java:do}
}

\newglossaryentry{java:do}
{
	type=java,
	name=do,
	description={Avainsana, joka luo do-while -loopin. Avainsanaa seuraa loopin runko
kaarisulkuihin suljettuna. Rungon jälkee määritellään loopin toistoehto while-avainsanalla, niin,
että avainsana kirjoitetaan rungon perään ja toistoehto suljetaan avainsanan jälkeen sulkuihin.
Toisin kuin while-looppi, joka tarkistaa toistoehdon ennen jokaista toistoa, do-while tarkistaa
toistoehdon jokaisen toiston jälkeen. Tukee continue- ja break-avainsanoja},
	see={java:while,java:continue,java:break}
}

\newglossaryentry{java:switch}
{
	type=java,
	name=switch,
	description={Avainsana, joka aloittaa valikkorakenteen. Avainsanaa seuraa valintaehto
sulkuihin suljettuna. Valikkorakenne koostuu case-avainsanalla määritellyistä haaroista ja
mahdollisesta default-avainsanalla määritellystä oletushaarasta. Koko valikkorakenne suljetaan
kaarisulkuihin. Valikkoehto lasketaan rakenteen alussa ja kaikki haarat, joiden arvo vastaa
laskettua ehtoa ajetaan järjestyksessä. Valikkorakenteesta voi poistua break-avainsanalla, mutta
se ei tue continue-avainsanaa. On normaalia päättää jokainen haara break-avainsanalla, jotta
haaroja ei ajettaisi vahingossa useampaa, ellei useamman haaran ajaminen ole
tarkoituksenmukaista},
	see={java:case,java:default,java:break}
}

\newglossaryentry{java:case}
{
	type=java,
	name=case,
	description={Avainsana, joka määrittää yksittäisen haaran switch-avainsanalla luodun
valikkorakenteen sisällä. Avainsanaa seuraa välilyönnillä erotettu haaran arvo, jota verrataan
switch-avainsanan määrittämään valintaehtoon. Haarat, joiden arvo toteutuu valintaehdossa ajetaan
järjestyksessä. On normaalia päättää haara break-avainsanaan, joka lopettaa koko rakenteen
ajamisen, ellei tavoitteena ole ajaa useampaa haaraa peräkkäin. Oletushaara, joka ajetaan, mikäli
mikään haara ei vastaa valintaehtoa voidaan määrittää default-avainsanalla},
	see={java:switch,java:break,java:default}
}

\newglossaryentry{java:default}
{
	type=java,
	name=default,
	description={Avainsana, joka määrittää oletushaaran switch-avainsanalla luodussa
valikkorakenteessa. Oletushaara ajetaan, jos valikkorakenteen valintaehto ei täsmää mitääm muuta
rakenteessa määriteltyä haaraa},
	see=java:switch
}

\newglossaryentry{java:break}
{
	type=java,
	name=break,
	description={Avainsana, joka lopettaa toistorakenteen toiston kokonaan tai poistuu
switch-rakenteesta. Käytettävissä kaikissa loopeissa ja switch-valintarakenteessa},
	see={java:for,java:while,java:do,java:switch}
}

\newglossaryentry{java:continue}
{
	type=java,
	name=continue,
	description={Avainsana, joka lopettaa toistorakenteen toiston ja aloittaa seuraavan toiston
saman tien. Käytettävissä kaikissa loopeissa},
	see={java:for,java:while,java:do}
}

\newglossaryentry{java:this}
{
	type=java,
	name=this,
	description={Avainsana, joka viittaa luokan instanssiin luokan omistaman instanssimetodin tai
luokan rakentajan sisällä. Ei toimi luokkametodeissa, koska niissä puuttuu instanssi, johon
viitata}
}

\newglossaryentry{java:return}
{
	type=java,
	name=return,
	description={Avainsana, joka aiheuttaa metodista poistumisen ja palauttaa avainsanaa seuraavan
arvon. Esimerkiksi "return 4", poistuisi metodista saman tien, palauttaen kokonaislukuarvon 4}
}

\newglossaryentry{java:final}
{
	type=java,
	name=final,
	description={Avainsana, joka määrittää muuttujjan olevan vakio, tai luokan olevan mahdoton
periyttää. Voi sijaita ennen tai jälkeen näkyvyysmääreen ja mahdollisen static-avainsanan.
Normaalisti viimeinen näistä kolmesta}
}

\newglossaryentry{java:enum}
{
	type=java,
	name=enum,
	description={Avainsana, joka määrittää luetellun tyypin. Korvaa class-avainsanan luokan
määrittelyssä. Luetellun tyypin määritelmä tarvitsee runkoonsa ainakin tyypin arvojoukon. Yleensä
tämän arvojoukon jäsenet kirjoitetaan CONSTANT\_CASE-tyylillä},
	see={lueteltu tyyppi,java:class}
}

\newglossaryentry{java:import}
{
	type=java,
	name=import,
	description={Avainsana, joka määrittää luokan tuotavaksi tiedoston nimitilaan}
}

\newglossaryentry{java:try}
{
	type=java,
	name=try,
	description={Avainsana, joka määrittää virheenkäsittelyrakenteen osion, joka sisältää
mahdollisen virhetilan aiheuttavan koodin. Try-osioita täytyy seurata joko yksi tai useampi 
catch-osio, finally-osio tai molemmat},
	see={java:catch,java:finally}
}

\newglossaryentry{java:catch}
{
	type=java,
	name=catch,
	description={Avainsana, joka määrittää virheenkäsittelyrakenteen virhetilankorjausosion
Try-avainsanalla aloitetussa virheenkäsittelyrakenteessa. Avainsanaa seuraa sulkuihin suljettuna
määritelmä muuttujalle, johon mahdollinen kiinni otettu virhe tallennetaan. Määritelmä määrittää
myös kiinni otettavan virheen tyypin. Jos virhe ei peri määriteltyä virhetyyppiä catch-rakenne ei
nappaa virhettä. Yhtä try-avainsanalla määriteltyä rakennetta kohden voi olla useampi
catch-rakenne, kunhan ne nappaavat erityyppisiä virheitä},
	see={java:try,java:finally}
}

\newglossaryentry{java:finally}
{
	type=java,
	name=finally,
	description={Avainsana, joka määrittää virheenkäsittelyrakenteen jaetun lopetusosion.
Finally-osio on aina try-avainsanalla aloitetun virheenkäsittelyrakenteen viimeinen osio ja se
ajetaan riippumatta siitä, aiheuttiko try-osion koodi virhetilan vai ei. Finally-osio sisältää
yleensä koodia, joka varmistaa ohjelmiston olevan jatkamiseen kelpaavassa tilassa
virheenkäsittelyn jälkeen}
}

\newglossaryentry{java:FileWriter}
{
	type=java,
	name=FileWriter,
	description={Standardikirjaston luokka, joka kirjoittaa merkkijonomuotoista dataa rakentajassa
määriteltyyn tiedostoon. Perii OutputStreamWriter- ja Writer -luokat ja sisältää niistä perityt
write- ja append -metodit kirjoitusoperaatioita varten}
}

\newglossaryentry{java:FileReader}
{
	type=java,
	name=FileReader,
	description={Standardikirjaston luokka, joka lukee merkkijonomuotoista dataa rakentajassa
määritellystä tiedostosta. Perii InputStreamReader- ja Reader -luokat ja sisältää niistä perityn
read-metodin lukuoperaatioita varten}
}

\newglossaryentry{java:FileInputStream}
{
	type=java,
	name=FileInputStream,
	description={Standardikirjaston luokka, joka sisältää tiedostoa edustavan syötteen tavuina
esitettynä virtana. Omistaa read-metodin syötteen lukemiseen. InputStream-luokan konkreettinen
lapsiluokka}
}

\newglossaryentry{java:FileOutputStream}
{
	type=java,
	name=FileOutputStream,
	description={Standardikirjaston luokka, joka sisältää tiedostoon kirjoitettavaa dataa
edustavan tavuina esitetyn virran. Omistaa write-metodin kirjoittamiseen. OutputStream-luokan
konkreettinen lapsiluokka},
	see={java:OutputStream,lapsiluokka}
}

\newglossaryentry{java:InputStream}
{
	type=java,
	name=InputStream,
	description={Abstrakti kantaluokka, joka lukee tavudataa lapsiluokassa määritellystä
lähteestä},
	see={abstrakti luokka,java:FileInputStream}
}

\newglossaryentry{java:OutputStream}
{
	type=java,
	name=OutputStream,
	description={Abstrakti kantaluokka, joka vastaanottaa tavudataa ja syöttää sen	
datankuluttajaan, joka määritellään lapsiluokassa},
	see={abstrakti luokka,java:FileOutputStream}
}

\newglossaryentry{java:extends}
{
	type=java,
	name=extends,
	description={Avainsana, joka määrittää luokan kantaluokan periytymisessä. Sijoitetaan luokan
nimen määritelmän perään niin, että avainsanaa seuraa kantaluokan nimi. Esimerkiksi luotaessa
luokkaa "ChildClass", joka halutaan periyttää kantaluokasta "ParentClass", kirjoitettaisiin luokan
määritelmä seuraavasti:\newline{}"public class ChildClass extends ParentClass..."\newline{}
Lisäksi avainsana toimii geneerisen luokan tai metodin tyyppiparametrin rajaamisessa syntaksilla
\newline{}"<T extends UpperBoundClass>"\newline{}Tällöin kaikkien tyyppien T pitää olla joko
UpperBoundClass luokan, tai siitä periytyvien luokkien instansseja},
	see={periytyminen,lapsiluokka,kantaluokka,tyyppiparametri}
}

\newglossaryentry{java:super}
{
	type=java,
	name=super,
	description={Avainsana, joka viittaa lapsiluokan kantaluokkaan instanssiin lapsiluokan
instanssin sisällä. Voidaan käyttää myös kantaluokan rakentajaan viittaamiseen, mikäli
avainsanaa käytetään funktiokutsuna ("super()" tai "super(args)")},
	see={periytyminen,lapsiluokka,kantaluokka,rakentaja}
}

\newglossaryentry{java:abstract}
{
	type=java,
	name=abstract,
	description={Avainsana, joka luo abstraktin luokan tai abstraktin metodin},
	see={abstrakti luokka,abstrakti metodi}
}

\newglossaryentry{java:Serializable}
{
	type=java,
	name=Serializable,
	description={Rajapintaluokka, joka määrittelee luokan olevan serialisoitavissa. Jotta luokka
voi implementoida Serializable-rajapinnan täytyy kaikkien luokan muuttujien olla myös
serialisoitavia. Suurin osa standardikirjaston luokista, sekä kaikki primitiiviset tietotyypit
ovat serialisoitavia, joten tämä rajoitus koskee lähinnä muuttujia, joiden tietotyyppi on jokin
käyttäjän määrittelemä luokka, tai jotka ovat geneerisiä käyttäjän määrittelemän luokan yli},
	see={primitiivinen tietotyyppi,geneerisyys}
}

\newglossaryentry{java:instanceof}
{
	type=java,
	name=instanceof,
	description={Avainsana, joka tarkistaa onko olio annetun luokan instanssi. Käyttö tapahtuu
muodossa "olio instanceof luokka"}
}

\newglossaryentry{java:ObjectOutputStream}
{
	type=java,
	name=ObjectOutputStream,
	description={Javan standardikirjaston luokka, jota käytetään Serializable-rajapinnan
implementoivien luokkien serialisaatioon. Luokan rakentaja ottaa vastaan OutputStream-luokan
konkreettisen lapsiluokan. Instanssista voidaan kutsua writeObject-metodia serialisoitavalla
oliolla, jolloin luokka kirjoittaa annetun olion serialisoidun tekstimuodon annettuun
OutputStream-olioon},
	see={java:OutputStream,java:Serializable,serialisaatio}
}

\newglossaryentry{java:ObjectInputStream}
{
	type=java,
	name=ObjectInputStream,
	description={Javan standardikirjaston luokka, jota käytetään Serializable-rajapinnan
implementoivien luokkien deserialisaatioon. Luokan rakentaja ottaa vastaan InputStream-luokan
konkreettisen lapsiluokan. Instanssista voidaan kutsua readObject-metodia, joka palauttaa
annetusta InputStream-oliosta luetusta tavudatasta deserialisoidun olion},
	see={java:InputStream,java:Serializable,serialisaatio}
}

\newglossaryentry{java:throw}
{
	type=java,
	name=throw,
	description={Avainsana, joka aiheuttaa virhetilan. Käytetään antamalla nostettava
virhetilainstanssi avainsana jälkeen. Esimerkiksi "throw new NullPointerException("oops")"
aiheuttaa NullPointerException-luokan virhetilan, jonka viestiksi on määritelty "oops"},
	see={java:try,java:catch,java:finally,java:throws}
}

\newglossaryentry{java:throws}
{
	type=java,
	name=throws,
	description={Avainsana, joka ilmoittaa metodin määritelmässä metodin voivan aiheuttaa jonkin
virhetilan. Avainsana sijaitsee metodin nimen ja parametrien määritelmän jälkeen ja
virhetilaluokka tai luokat listataan avainsanan jälkeen. Esimerkiksi "public void doStuff(String
arg) throws NullPointerException" luo metodin nimeltä doStuff, joka ilmoittaa voivansa aiheuttaa
NullPointerException-luokan virhetilan},
	see={java:try,java:catch,java:finally,java:throw,metodi}
}

\newglossaryentry{java:Exception}
{
	type=java,
	name=Exception,
	description={Standardikirjaston luokka, joka toimii kaikkien korjattavissa olevien
virhetilojen kantaluokkana}
}

\newglossaryentry{java:Iterator}
{
	type=java,
	name=Iterator,
	description={Standardikirjaston rajapintaluokka, joka mahdollistaa iteroinnin jonkin kokoelman
yli. Luodaan kutsumalla iterator-metodia iteroitavasta kokoelmasta. Tarjoaa metodit next, hasNext,
remove ja forEachRemaining},
	see={java:iterator,java:ListIterator,java:listIterator}
}

\newglossaryentry{java:iterator}
{
	type=java,
	name=iterator,
	description={Standardikirjaton kokoelmien omistama metodi, joka luo Iterator-olion kyseisestä
kokoelmasta},
	see={java:Iterator}
}

\newglossaryentry{java:ListIterator}
{
	type=java,
	name=ListIterator,
	description={Standardikirjaston rajapintaluokka, joka mahdollistaa iteroinnin jonkin
List-rajapinnan implementoivan kokoelman yli. Rajapinnan implementoivan luokan instanssi saadaan
kutsumalla listIterator metodia iteroitavasta List-instanssista. Rajapinta tarjoaa metodit, next,
hasNext, nextIndex, previous, hasPrevious, previousIndex, remove, add(E element) ja set(E
element)},
	see={java:listIterator,java:Iterator,java:iterator}
}

\newglossaryentry{java:listIterator}
{
	type=java,
	name=iterator,
	description={Standardikirjaston List-rajapintaluokan olioiden omistama metodi, joka luo
ListIterator-olion kyseisestä List-instanssista},
	see={java:ListIterator,java:List}
}

\newglossaryentry{java:Consumer}
{
	type=java,
	name=Consumer,
	description={Standardikirjaston rajapintaluokka, joka määrittelee niin sanotun kuluttajan.
Kuluttaja on geneerinen olio, joka on parametrisoitu tyypillä T ja jolla on yksi metodi, joka ei
palauta mitään ja ottaa argumentiksi yhden tyypin T instanssin. Tämän metodin nimi rajapinnassa on
accept. Rajapinta tarjoaa lisäksi andThen-metodin, joka ottaa toisen kuluttajan, jonka täytyy myös
olla parametrisoitu tyypillä T. Metodi yhdistää kuluttajien kutsut ja palauttaa kutsut yhdistävän
kuluttajainstanssin, joka kutsuu vuorotellen kummankin kuluttajan accept-metodit},
	see={kuluttaja,java:interface,geneerinen luokka,tyyppiparametri}
}

\newglossaryentry{java:Object}
{
	type=java,
	name=Object,
	description={Standardikirjaston luokka, josta kaikki muut Javan oliot periytyvät. Sisältää
fundamentaalisia, kaikille olioille saatavilla olevia metodeja, kuten hashCode, jota käytetään
HashMap-luokkaa luotaessa ja equals(Object), jota käytetään vertailussa},
	see=java:HashMap
}

\newglossaryentry{java:Clonable}
{
	type=java,
	name=Clonable,
	description={Standardikirjaston rajapintaluokka, joka määrittää implementoivan olion olevan
kloonattavissa clone-metodilla. Rajapinnan implementoivan luokan on ylikirjoitettava Object-luokan
määrittelemä clone-metodi},
	see={rajapinta,java:Object}
}