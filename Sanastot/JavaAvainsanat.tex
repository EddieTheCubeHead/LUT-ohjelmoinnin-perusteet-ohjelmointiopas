\newglossaryentry{java:interface}
{
    type=java,
    name=interface,
    description={Luo abstraktin rajapinnan, jonka luokka voi implementoida. Näin luotu
rajapintapohja ei voi sisältää datakenttiä eikä sitä voi käyttää olion muottina. Sen sijaan luokka
voi implementoida rajapinnan jolloin luokka lupaa muulle ohjelmistolle tarjoavansa rajapinnan
määrittelemät funktiot}
}

\newglossaryentry{java:bool}
{
    type=java,
    name=bool,
    description={Primitiivinen tietotyyppi, joka sisältää totuusarvon (true tai false)},
    see={primitiivinen tietotyyppi}
}

\newglossaryentry{java:byte}
{
    type=java,
    name=byte,
    description={Primitiivinen tietotyyppi, joka sisältää tavun kokoisen merkillisen
kokonaisluvun},
    see={primitiivinen tietotyyppi}
}

\newglossaryentry{java:short}
{
    type=java,
    name=short,
    description={Primitiivinen tietotyyppi, joka sisältää kahden tavun kokoisen merkillisen
kokonaisluvun},
    see={primitiivinen tietotyyppi}
}

\newglossaryentry{java:int}
{
    type=java,
    name=int,
    description={Primitiivinen tietotyyppi, joka sisältää 32 bitin kokoisen merkillisen
kokonaisluvun},
    see={primitiivinen tietotyyppi}
}

\newglossaryentry{java:long}
{
    type=java,
    name=long,
    description={Primitiivinen tietotyyppi, joka sisältää 64 bitin kokoisen merkillisen
kokonaisluvun},
    see={primitiivinen tietotyyppi}
}

\newglossaryentry{java:float}
{
    type=java,
    name=float,
    description={Primitiivinen tietotyyppi, joka sisältää 32 bitin kokoisen liukumaesitetyn
desimaaliluvun},
    see={primitiivinen tietotyyppi}
}

\newglossaryentry{java:double}
{
    type=java,
    name=double,
    description={Primitiivinen tietotyyppi, joka sisältää 64 bitin kokoisen liukumaesitetyn
desimaaliluvun},
    see={primitiivinen tietotyyppi}
}

\newglossaryentry{java:char}
{
    type=java,
    name=char,
    description={Primitiivinen tietotyyppi, joka sisältää kahden tavun kokoisen unicode-koodatun
merkin esitettynä merkittömänä kokonaislukuna},
    see={primitiivinen tietotyyppi}
}

\newglossaryentry{java:String}
{
    type=java,
    name=String,
    description={Javan standardikirjaston merkkijonoimplementaatioluokka. Suositellaan
käytettäväksi merkkijonojen säilömiseen koodissa. Pystyy säilömään dynaamisen merkkijonon, jonka
alustus- tai maksimikokoa ei tarvitse määrittää erikseen. Lisäksi sisältää lukuisia merkkijonon
käsittelyä ja muokkausta helpottavia metodeja}
}

\newglossaryentry{java:private}
{
	type=java,
	name=private,
	description={Näkyvyysmääre, joka määrittää ominaisuuden olevan käytettävissä vain ominaisuuden
omistaman luokan sisällä}
}

\newglossaryentry{java:protected}
{
	type=java,
	name=protected,
	description={Näkyvyysmääre, joka määrittää ominaisuuden olevan käytettävissä ominaisuuden
omistavan luokan sisältävässä packagessa ja kaikissa ominaisuuden omistavan luokan perivissä
luokissa}
}

\newglossaryentry{java:public}
{
	type=java,
	name=public,
	description={Näkyvyysmääre, joka määrittää ominaisuuden olevan käytettävissä kaikkialla
ohjelmistossa}
}

\newglossaryentry{java:static}
{
	type=java,
	name=static,
	description={Staattinen metodi tai muuttuja näkyy kaikille sen omistavan luokan instansseille
jaetusti. Tunnetaan luokkamuuttujana tai luokkametodina}
}

\newglossaryentry{java:void}
{
	type=java,
	name=void,
	description={Muuttujan paluuarvotyyppi muuttujalle, joka ei palauta mitään}
}

\newglossaryentry{java:main}
{
	type=java,
	name=main,
	description={Varattu metodinimi metodille, jonka Java ajaa ensimmäisenä ajaessaan ohjelmistoa.
Ohjelmistossa voi olla useampi-main niminen metodi, mutta vain määritellyn juuriluokan main-metodi
ajetaan. Metodin täytyy olla muotoa public static void ja ottaa yhden taulukon String-luokan
instansseja}
}

\newglossaryentry{java:class}
{
	type=java,
	name=class,
	description={Avainsana, joka aloittaa luokan määritelmän.}
}

\newglossaryentry{java:if}
{
	type=java,
	name=if,
	description={Avainsana, jota seuraa normaaleihin sulkuihin suljettu totuusarvo ja
kaarisulkuihin suljettu koodin osa. Tämä koodin osa ajetaan vain jos annettu totuusarvo on
tosi. Voidaan yhdistää else-lauseeseen},
	see=java:else
}

\newglossaryentry{java:else}
{
	type=java,
	name=else,
	description={Avainsana, joka voi seurata if-lauseella määriteltyä ehdollista koodin osaa.
Else-lausetta seuraa kaarisulkuihin suljettu koodin osa. Tämä koodin osa ajetaan vain, jos
else-lausetta edeltänyttä ehdollista koodin osaa ei ajettu},
	see=java:if
}

\newglossaryentry{java:new}
{
	type=java,
	name=new,
	description={Avainsana, joka luo uuden instanssin luokasta. Avainsanaa seuraa aina toivotun
luokan nimi, jonka perässä on sulkuihin suljettuna toivotut rakentajalle annettavat parametrit,
metodikutsun tapaan.},
	see=rakentaja
}