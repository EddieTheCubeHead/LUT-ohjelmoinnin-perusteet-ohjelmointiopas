\newglossaryentry{abstraktio}
{
    name=abstraktio,
    description={(abstraction) Ohjelmoinnin perustekniikka, jossa ongelman
tarkka ratkaisu piilotetaan kutsuttavan koodirakenteen, kuten funktion, 
tietorakenteen tai luokan taakse}
}

\newglossaryentry{funktio}
{
    name=funktio,
    description={(function) Ohjelmoijan määrittelemä käskysarja, eli koodin
osa, joka on rajattu, ottaa tietyn määrän parametreja ja mahdollisesti
palauttaa paluuarvon. Tunnetaan olio-ohjelmoinnissa nimellä metodi},
    see={parametri,metodi}
}

\newglossaryentry{parametri}
{
    name=parametri,
    description={(parameter) Arvo, jonka funktio tai metodi ottaa muulta
koodilta vastaan},
    see={funktio,metodi}
}

\newglossaryentry{metodi}
{
    name=metodi,
    description={(method) Luokkaan sidottu käskysarja, joka suorittaa
ottamiensa parametrien ja luokan omien datakenttien perusteella jonkin tietyn
toiminnon},
    see={parametri,funktio,luokka}
}

\newglossaryentry{tietue}
{
    name=tietue,
    description={(struct) Vanhahko muokattava tietotyyppi, yleinen esimerkiksi
C-kielessä. Käyttäjä voi määritellä tietueen sisältämään mitä tahansa
vakiokokoisia datakenttiä. Luokkien edeltäjä}
}

\newglossaryentry{olio}
{
    name=olio,
    description={(object) Luokan instanssi. Yksittäinen koodissa luotu
toimija, joka sisältää datakenttiä ja metodeita. Luokka, jonka pohjalta olio
luodaan määrittää olion käytettävissä olevat metodit ja siihen
tallennetut datatyypit, mutta vain olio pääsee käsiksi omiin metodeihin
ja datakenttiinsä},
    see={luokka,metodi,instanssi}
}

\newglossaryentry{luokka}
{
    name=luokka,
    description={(class) Ohjelmoijan kirjoittama muotti, jonka pohjalta
ohjelmisto luo olioita. Voi sisältää metodeja ja datakenttiä mutta yleensä
näiden käyttämiseksi vaaditaan olion luontia},
    see={olio,metodi,instanssi}
}

\newglossaryentry{instanssi}
{
    name=instanssi,
    description={(instance) Olio, joka on luotu jonkin luokan pohjalta on
kyseisen luokan instanssi},
    see={olio,luokka}
}

\newglossaryentry{enkapsulaatio}
{
    name=enkapsulaatio,
    description={(encapsulation) Datan piilottaminen olion sisään niin ettei
muu ohjelmisto näe kyseistä dataa. Mitataan asteikolla matala-korkea, niin että
korkea enkapsulaatio tarkoittaa pientä määrää julkisia metodeja tai datakenttiä
ja matala taas suurta määrää julkisia metodeja ja datakenttiä}
    see={luokka,koheesio,metodi}
}

\newglossaryentry{rajapinta}
{
    name=rajapinta,
    description={(interface) Termi ohjelmistossa toteutuvalle sopimukselle,
jonka jokin metodi, luokka tmv. toteuttaa. Myös kahden ohjelmiston osan välinen
taso. Myös Javan Interface-avainsana},
    see={java:interface}
}

\newglossaryentry{koheesio}
{
    name=koheesio,
    description={(cohesion) Ohjelmiston laadun mittaamiseen käytetty käsite.
Mittaa luokkien sisäistä yhtenäisyyttä akselilla matala-korkea. Matala koheesio
tarkoittaa että luokassa on paljon metodeja jotka eivät keskustele toisten
luokan metodien kanssa ja matala että luokan kaikki metodit käyttävät useita
muita luokan metodeja. Matala koheesio on toivottavaa, koska luokan tehtävä
on tehdä yksi ja vain yksi asia}
}

\newglossaryentry{pariutuminen}
{
    name=pariutuminen,
    description={(coupling) Ohjelman laadun mittaamiseen käytetty käsite.
Mittaa luokkien keskenäisten riippuvuuksien määrää akselilla löysä-tiukka.
Löysässä pariutumisessa ohjelmiston luokkien väliset riippuvuudet ovat
harvassa, jolloin ohjelmiston muokkaaminen on helppoa. Tiukassa pariutumisessa
puolestaan jokaisella ohjelmiston luokalla on riippuvuus moneen muuhun
ohjelmiston luokkaan, jolloin ohjelmison muokkaus hankaloituu ja täten voidaan
katsoa ohjelmiston laadun laskevan}
}

\newglossaryentry{primitiivinen tietotyyppi}{
    name=primitiivinen tietotyyppi,
    description={(primitive data type) Tietotyyppi, jonka ohjelmointikieli
kykenee säilömään suoraan muistipaikkaan raakana numeerisena datana. Ainoat
tietotyypit Javassa, jotka eivät ole jonkin luokan instansseja},
    see={luokka,instanssi}
}

\newglossaryentry{nakyvyysmaare}
{
    name=näkyvyysmääre,
    description={(access modifier) Muuttujan näkyvyyden määrittävä avainsana},
    see={muuttuja}
}

\newglossaryentry{muuttuja}
{
    name=muuttuja,
    description={(variable) koodissa määritelty tietokenttä, joka sisältää jonkin ohjelman
käyttämän arvon}
}

\newglossaryentry{staattinen kieli}
{
    name=staattisesti tyypitetty kieli,
    description={(statically typed language) Ohjelmointikieli, joka tietää jokaisen koodissa
esiintyvän muuttujan tietotyypin koko ajan}
}

\newglossaryentry{signatuuri}
{
	name=signatuuri,
	description={(signature) Metodin määritelmä, joka sisältää sekä metodin nimen, että sen
ottamien argumenttien tyypit.}
}