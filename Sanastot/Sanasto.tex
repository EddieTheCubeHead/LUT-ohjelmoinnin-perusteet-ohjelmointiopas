\newglossaryentry{abstraktio}
{
    name=abstraktio,
    description={(abstraction) Ohjelmoinnin perustekniikka, jossa ongelman
tarkka ratkaisu piilotetaan kutsuttavan koodirakenteen, kuten funktion, 
tietorakenteen tai luokan taakse}
}

\newglossaryentry{funktio}
{
    name=funktio,
    description={(function) Ohjelmoijan määrittelemä käskysarja, eli koodin
osa, joka on rajattu, ottaa tietyn määrän parametreja ja mahdollisesti
palauttaa paluuarvon. Tunnetaan olio-ohjelmoinnissa nimellä metodi},
    see={parametri,metodi}
}

\newglossaryentry{parametri}
{
    name=parametri,
    description={(parameter) Arvo, jonka funktio tai metodi ottaa muulta
koodilta vastaan},
    see={funktio,metodi}
}

\newglossaryentry{metodi}
{
    name=metodi,
    description={(method) Luokkaan sidottu käskysarja, joka suorittaa
ottamiensa parametrien ja luokan omien datakenttien perusteella jonkin tietyn
toiminnon},
    see={parametri,funktio,luokka}
}

\newglossaryentry{tietue}
{
    name=tietue,
    description={(struct) Vanhahko muokattava tietotyyppi, yleinen esimerkiksi
C-kielessä. Käyttäjä voi määritellä tietueen sisältämään mitä tahansa
vakiokokoisia datakenttiä. Luokkien edeltäjä}
}

\newglossaryentry{olio}
{
    name=olio,
    description={(object) Luokan instanssi. Yksittäinen koodissa luotu
toimija, joka sisältää datakenttiä ja metodeita. Luokka, jonka pohjalta olio
luodaan määrittää olion käytettävissä olevat metodit ja siihen
tallennetut datatyypit, mutta vain olio pääsee käsiksi omiin metodeihin
ja datakenttiinsä},
    see={luokka,metodi,instanssi}
}

\newglossaryentry{luokka}
{
    name=luokka,
    description={(class) Ohjelmoijan kirjoittama muotti, jonka pohjalta
ohjelmisto luo olioita. Voi sisältää metodeja ja datakenttiä mutta yleensä
näiden käyttämiseksi vaaditaan olion luontia},
    see={olio,metodi,instanssi}
}

\newglossaryentry{instanssi}
{
    name=instanssi,
    description={(instance) Olio, joka on luotu jonkin luokan pohjalta on
kyseisen luokan instanssi},
    see={olio,luokka}
}

\newglossaryentry{enkapsulaatio}
{
    name=enkapsulaatio,
    description={(encapsulation) Datan piilottaminen olion sisään niin ettei
muu ohjelmisto näe kyseistä dataa. Mitataan asteikolla matala-korkea, niin että
korkea enkapsulaatio tarkoittaa pientä määrää julkisia metodeja tai datakenttiä
ja matala taas suurta määrää julkisia metodeja ja datakenttiä}
    see={luokka,koheesio,metodi}
}

\newglossaryentry{rajapinta}
{
    name=rajapinta,
    description={(interface) Yleisessä käytössä termi ohjelmistossa toteutuvalle sopimukselle,
jonka jokin metodi, luokka tmv. toteuttaa tai kahden ohjelmiston osan välinen taso. Javan
yhteydessä luokka, joka luodaan interface-avainsanalla ja joka sisältää vain
luokan implementoimien metodien signatuurit. Rajapintaluokka ei yleensä sisällä ollenkaan
toiminnallista koodia},
    see={java:interface}
}

\newglossaryentry{koheesio}
{
    name=koheesio,
    description={(cohesion) Ohjelmiston laadun mittaamiseen käytetty käsite.
Mittaa luokkien sisäistä yhtenäisyyttä akselilla matala-korkea. Matala koheesio
tarkoittaa että luokassa on paljon metodeja jotka eivät keskustele toisten
luokan metodien kanssa ja matala että luokan kaikki metodit käyttävät useita
muita luokan metodeja. Matala koheesio on toivottavaa, koska luokan tehtävä
on tehdä yksi ja vain yksi asia}
}

\newglossaryentry{pariutuminen}
{
    name=pariutuminen,
    description={(coupling) Ohjelman laadun mittaamiseen käytetty käsite.
Mittaa luokkien keskenäisten riippuvuuksien määrää akselilla löysä-tiukka.
Löysässä pariutumisessa ohjelmiston luokkien väliset riippuvuudet ovat
harvassa, jolloin ohjelmiston muokkaaminen on helppoa. Tiukassa pariutumisessa
puolestaan jokaisella ohjelmiston luokalla on riippuvuus moneen muuhun
ohjelmiston luokkaan, jolloin ohjelmison muokkaus hankaloituu ja täten voidaan
katsoa ohjelmiston laadun laskevan}
}

\newglossaryentry{primitiivinen tietotyyppi}{
    name=primitiivinen tietotyyppi,
    description={(primitive data type) Tietotyyppi, jonka ohjelmointikieli
kykenee säilömään suoraan muistipaikkaan raakana numeerisena datana. Ainoat
tietotyypit Javassa, jotka eivät ole jonkin luokan instansseja},
    see={luokka,instanssi}
}

\newglossaryentry{nakyvyysmaare}
{
    name=näkyvyysmääre,
    description={(access modifier) Muuttujan näkyvyyden määrittävä avainsana},
    see=muuttuja
}

\newglossaryentry{muuttuja}
{
    name=muuttuja,
    description={(variable) koodissa määritelty tietokenttä, joka sisältää jonkin ohjelman
käyttämän arvon}
}

\newglossaryentry{staattinen kieli}
{
    name=staattisesti tyypitetty kieli,
    description={(statically typed language) Ohjelmointikieli, joka tietää jokaisen koodissa
esiintyvän muuttujan tietotyypin koko ajan}
}

\newglossaryentry{signatuuri}
{
	name=signatuuri,
	description={(signature) Metodin määritelmä, joka sisältää sekä metodin nimen, että sen
ottamien argumenttien tyypit}
}

\newglossaryentry{noutaja}
{
	name=noutaja,
	description={(getter) Metodi, jonka tehtävänä on tarjota rajoitettu saatavuus johonkin
luokan yksityiseen muuttujan. Ei yleensä ota argumentteja ja palauttaa luokan omistaman toivotun
muuttujan. Nimetään camelCase-tyylillä sijoittamalla muuttujan nimen eteen "get" ("getVariable").
Parantaa enkapsulaatiota ja mahdollistaa esimerkiksi laiskan alustuksen muuttujille}
}

\newglossaryentry{asettaja}
{
	name=asettaja,
	description={(setter) Metodi, jonka tehtävänä on tarjota luokan ulkopuolisille toimijoille
mahdollisuus asettaa arvoja luokan yksityiseen muuttujaan. Ottaa yleensä arvoksi muuttujan toivotun
arvon, eikä palauta mitään. Nimetään camelCase-tyylillä sijoittamalla muuttujan nimen eteen "set"
("setVariable"). Parantaa enkapsulaatiota ja mahdollistaa esimerkiksi uusien arvojen validoinnin
ennen niiden asettamista}
}

\newglossaryentry{rakentaja}
{
	name=rakentaja,
	description={(constructor) Luokkametodi, joka määritellään ja kutsutaan luokan nimellä ja nimensä
mukaisesti rakentaa ja palauttaa uuden luokan instanssin},
	see={luokkametodi,instanssi,java:new}
}

\newglossaryentry{luokkametodi}
{
	name=luokkametodi,
	description={(class method) Metodi, joka on luokan instanssin sijaan sidottu itse luokkaan.
Ei voida käyttää instanssiin sidottuja muuttujia, mutta voi käyttää luokkamuuttujia. Määritellään
yleensä static-avainsanalla ja kutsutaan luokan nimen pistenotaatiolla},
	see={luokkamuuttuja,java:static}
}

\newglossaryentry{luokkamuuttuja}
{
	name=luokkamuuttuja,
	description={(class variable) Muuttuja, joka on luokan instanssin sijaan sidottu itse luokkaan.
Muokattavissa ja tarkasteltavissa jokaisesta luokan instanssista jaetusti. Määritellään yleensä
static-avainsanalla},
	see=java:static
}

\newglossaryentry{taulukko}
{
	name=taulukko,
	description={(array) Tietorakenne, jossa yksittäiset alkiot on säilötty vierekkäisiin
muistipaikkoihin. Vie vähän muistitilaa ja mahdollistaa nopeat haut ja lisäykset jos alkion
indeksi on tiedossa etukäteen. Yleensä taulukon koko on määriteltävä sen luomisen yhteydessä,
tämä on totta myös Javassa. Javassa taulukko luodaan new-avainsanalla lisäämällä taulukon
säilömän tietotyypin perään hakasulkeet, joiden sisään suljetaan taulukon koon määrittelevä
numero. Taulukkotyyppinen muuttuja määritellään lisäämällä muuttujan tietotyypin määritelmän
perään hakasulkeet. Tällöin esimerkiksi int-tyypin kymmenen alkion taulukko, joka on säilötty
"intArray"-muuttujaan määriteltäisiin seuraavasti:\newline{} "int[] intArray = new int[10]"}
}

\newglossaryentry{geneerinen luokka}
{
	name=geneerinen luokka,
	description={(generic class) Luokan ominaisuus, joka määrittää luokan instanssien täyden
tyyppisignatuurin olevan riippuvainen toisesta luokasta. Tämä toinen luokka voidaan määrittää
jokaisen luokan instanssin kohdalla erikseen pienempi kuin- ja suurempi kuin -merkkien väliin
suljetulla notaatiolla. Esimerkiksi kokoelmat tarvitsevat tietoa sisältämiensä olioiden
tyypityksestä. Merkkijonoja sisältävä ArrayList-kokoelma voitaisiin siis luoda seuraavalla 
notaatiolla:\newline{}"new ArrayList<String>()"\newline{}Geneerinen luokka voi olla myös geneerinen useamman luokan suhteen, jolloin luokat erotetaan pilkulla tai riippuvainen toisesta
geneerisestä luokasta, jolloin geneerisyyden notaatiot kirjoitetaan
sisäkkäin:\newline{}"new HashMap<String, ArrayList<Int>>()"},
	see={java:ArrayList,java:HashMap}
}

\newglossaryentry{hajautustaulu}
{
	name=hajautustaulu,
	description={(hash table) Tietorakenne, joka koostuu avain-arvo -pareista. Jokaista avainta
vastaa yksi arvo. Arvot ovat noudettavissa nopeasti avaimen perusteella. Toteutettu Javan
standardikirjaston HashMap-luokassa},
	see={java:HashMap}
}

\newglossaryentry{ajonaikainen kaantaminen}
{
	name=ajonaikainen kaantaminen,
	description={(Just-In-Time compiling, JIT) Ajonaikainen kääntäminen on koodin ajotekniikka,
jossa}
}


\newglossaryentry{kaannetty kieli}
{
	name=kaannetty kieli,
	description={(compiled language) Ohjelmointikieli, jossa ohjelmistot kompiloidaan ennen
ajamista ja ohjelman ajaminen tapahtuu suorittamalla kompilointiprosessin tuottama tiedosto.
Esimerkiksi c ja Rust ovat kompiloituja kieliä}
}


\newglossaryentry{tulkattu kieli}
{
	name=tulkattu kieli,
	description={(intrepeted language) Ohjelmointikieli, joka ajetaan lukemalla kooditiedostot
reaaliajassa. Esimerkiksi Python ja JavaScript ovat tulkattavia kieliä}
}

\newglossaryentry{roskankeruu}
{
	name=roskankeruu,
	description={(garbage collection) Muistinhallinnan tekniikka, jossa ohjelma tietyin väliajoin
ajaa aliohjelman, joka käy läpi kaikki muuttujat muistissa, tarkistaa onko niihin olemassa
viittauksia ja poistaa muuttujat, joilla ei ole enää aktiivista viittausta}
}

\newglossaryentry{ylikuormitus}
{
	name=ylikuormitus,
	description={(overloading) Kahden samannimisen metodin määrittäminen eri argumenteilla.
Argumenttejä voi olla sama määrä, mutta eri tyypeillä, tai eri määrä. Java hakee metodikutsun
yhteydessä automaattisesti kutsuttua signatuuria vastaavan version ylikuormitetusta metodista}
}