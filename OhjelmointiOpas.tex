\documentclass{tufte-book}

\hypersetup{colorlinks}

%%
% Omat komennot
\newcommand{\lut}{Lappeenrannan-Lahden teknillinen yliopisto LUT }
\newcommand{\lutin}{Lappeenrannan-Lahden teknillisen yliopiston LUT:n }
\newcommand{\eng}[1]{\textit{(#1)}}
\newcommand{\new}[1]{\textit{\gls{#1}}}
\newcommand{\newnogls}[1]{\textit{#1}}
\newcommand{\neweng}[2]{\new{#1} \eng{#2}}
\newcommand{\newengnogls}[2]{\newnogls{#1} \eng{#2}}
\newcommand{\java}[1]{\underline{\gls{java:#1}}}
\newcommand{\javanogls}[1]{\underline{#1}}
\newcommand{\newjava}[1]{\textit{\java{#1}}}
\newcommand{\newjavanogls}[1]{\textit{\javanogls{#1}}}


%%
% Book metadata
\title{Ohjelmointiopas{\small \lutin olio-ohjelmointikurssi}\thanks{Kiitokset kandityön valvojalle
Erno Vanhalalle.}}
\author[Eetu Juhani Asikainen]{Eetu Juhani Asikainen}
\publisher{Lisää julkaisija?}

%%
% For nicely typeset tabular material
\usepackage{booktabs}

%%
% Prints a trailing space in a smart way.
\usepackage{xspace}

% Prints the month name (e.g., January) and the year (e.g., 2008)
\newcommand{\monthyear}{
  \ifcase\month\or January\or February\or March\or April\or May\or June\or
  July\or August\or September\or October\or November\or
  December\fi\space\number\year
}

\usepackage{units}

% Generates the index
\usepackage{makeidx}
\makeindex

\usepackage[toc, automake]{glossaries}
\newglossary[jvg]{java}{jvd}{jvn}{Javan avainsanat}
\glsaddall
\makeglossaries

%%%%%%%%%%%%%%%%%%%%%%%%%%%%%%%%%%%%%%%%%%%%%%%%%%%%%%%%%%%%%%%%%%%%%%%%%%%%%%%%%%%%%%%%%%%%%%%%%%
% % % % % % % % % % % % % % % % % % % % % % % % % % % % % % % % % % % % % % % % % % % % % % % % %
%%%%%%%%%%%%%%%%%%%%%%%%%%%%%%%%%%%%%%%%%%%%%%%%%%%%%%%%%%%%%%%%%%%%%%%%%%%%%%%%%%%%%%%%%%%%%%%%%%

%% Java code highlighting
\usepackage[utf8]{inputenc}
\usepackage{pygmentex}
\usepackage[cache=false]{minted} 
\usepackage{listings}
\renewcommand{\lstlistingname}{Koodiesimerkki}
\renewcommand{\lstlistlistingname}{Koodiesimerkit}

\newcommand{\code}[3]{
\begin{listing}
    \inputminted{java}{OhjelmointiopasEsimerkit/src/#1/#2.java}
    \caption{#3}
    \label{Java-#1-#2}
\end{listing}
}

%%%%%%%%%%%%%%%%%%%%%%%%%%%%%%%%%%%%%%%%%%%%%%%%%%%%%%%%%%%%%%%%%%%%%%%%%%%%%%%%%%%%%%%%%%%%%%%%%%
% % % % % % % % % % % % % % % % % % % % % % % % % % % % % % % % % % % % % % % % % % % % % % % % %
%%%%%%%%%%%%%%%%%%%%%%%%%%%%%%%%%%%%%%%%%%%%%%%%%%%%%%%%%%%%%%%%%%%%%%%%%%%%%%%%%%%%%%%%%%%%%%%%%%

%% SANASTOT
\newglossaryentry{abstraktio}
{
    name=abstraktio,
    description={(abstraction) Ohjelmoinnin perustekniikka, jossa ongelman
tarkka ratkaisu piilotetaan kutsuttavan koodirakenteen, kuten funktion, 
tietorakenteen tai luokan taakse}
}

\newglossaryentry{funktio}
{
    name=funktio,
    description={(function) Ohjelmoijan määrittelemä käskysarja, eli koodin
osa, joka on rajattu, ottaa tietyn määrän parametreja ja mahdollisesti
palauttaa paluuarvon. Tunnetaan olio-ohjelmoinnissa nimellä metodi},
    see={parametri,metodi}
}

\newglossaryentry{parametri}
{
    name=parametri,
    description={(parameter) Arvo, jonka funktio tai metodi ottaa muulta
koodilta vastaan},
    see={funktio,metodi}
}

\newglossaryentry{metodi}
{
    name=metodi,
    description={(method) Luokkaan sidottu käskysarja, joka suorittaa
ottamiensa parametrien ja luokan omien datakenttien perusteella jonkin tietyn
toiminnon},
    see={parametri,funktio,luokka}
}

\newglossaryentry{tietue}
{
    name=tietue,
    description={(struct) Vanhahko muokattava tietotyyppi, yleinen esimerkiksi
C-kielessä. Käyttäjä voi määritellä tietueen sisältämään mitä tahansa
vakiokokoisia datakenttiä. Luokkien edeltäjä}
}

\newglossaryentry{olio}
{
    name=olio,
    description={(object) Luokan instanssi. Yksittäinen koodissa luotu
toimija, joka sisältää datakenttiä ja metodeita. Luokka, jonka pohjalta olio
luodaan määrittää olion käytettävissä olevat metodit ja siihen
tallennetut datatyypit, mutta vain olio pääsee käsiksi omiin metodeihin
ja datakenttiinsä},
    see={luokka,metodi,instanssi}
}

\newglossaryentry{luokka}
{
    name=luokka,
    description={(class) Ohjelmoijan kirjoittama muotti, jonka pohjalta
ohjelmisto luo olioita. Voi sisältää metodeja ja datakenttiä mutta yleensä
näiden käyttämiseksi vaaditaan olion luontia},
    see={olio,metodi,instanssi}
}

\newglossaryentry{instanssi}
{
    name=instanssi,
    description={(instance) Olio, joka on luotu jonkin luokan pohjalta on
kyseisen luokan instanssi},
    see={olio,luokka}
}

\newglossaryentry{enkapsulaatio}
{
    name=enkapsulaatio,
    description={(encapsulation) Datan piilottaminen olion sisään niin ettei
muu ohjelmisto näe kyseistä dataa. Mitataan asteikolla matala-korkea, niin että
korkea enkapsulaatio tarkoittaa pientä määrää julkisia metodeja tai datakenttiä
ja matala taas suurta määrää julkisia metodeja ja datakenttiä}
    see={luokka,koheesio,metodi}
}

\newglossaryentry{rajapinta}
{
    name=rajapinta,
    description={(interface) Yleisessä käytössä termi ohjelmistossa toteutuvalle sopimukselle,
jonka jokin metodi, luokka tmv. toteuttaa tai kahden ohjelmiston osan välinen taso. Javan
yhteydessä luokka, joka luodaan interface-avainsanalla ja joka sisältää vain
luokan implementoimien metodien signatuurit. Rajapintaluokka ei yleensä sisällä ollenkaan
toiminnallista koodia},
    see={java:interface}
}

\newglossaryentry{koheesio}
{
    name=koheesio,
    description={(cohesion) Ohjelmiston laadun mittaamiseen käytetty käsite.
Mittaa luokkien sisäistä yhtenäisyyttä akselilla matala-korkea. Matala koheesio
tarkoittaa että luokassa on paljon metodeja jotka eivät keskustele toisten
luokan metodien kanssa ja matala että luokan kaikki metodit käyttävät useita
muita luokan metodeja. Matala koheesio on toivottavaa, koska luokan tehtävä
on tehdä yksi ja vain yksi asia}
}

\newglossaryentry{pariutuminen}
{
    name=pariutuminen,
    description={(coupling) Ohjelman laadun mittaamiseen käytetty käsite.
Mittaa luokkien keskenäisten riippuvuuksien määrää akselilla löysä-tiukka.
Löysässä pariutumisessa ohjelmiston luokkien väliset riippuvuudet ovat
harvassa, jolloin ohjelmiston muokkaaminen on helppoa. Tiukassa pariutumisessa
puolestaan jokaisella ohjelmiston luokalla on riippuvuus moneen muuhun
ohjelmiston luokkaan, jolloin ohjelmison muokkaus hankaloituu ja täten voidaan
katsoa ohjelmiston laadun laskevan}
}

\newglossaryentry{primitiivinen tietotyyppi}{
    name=primitiivinen tietotyyppi,
    description={(primitive data type) Tietotyyppi, jonka ohjelmointikieli
kykenee säilömään suoraan muistipaikkaan raakana numeerisena datana. Ainoat
tietotyypit Javassa, jotka eivät ole jonkin luokan instansseja},
    see={luokka,instanssi}
}

\newglossaryentry{nakyvyysmaare}
{
    name=näkyvyysmääre,
    description={(access modifier) Muuttujan näkyvyyden määrittävä avainsana},
    see=muuttuja
}

\newglossaryentry{muuttuja}
{
    name=muuttuja,
    description={(variable) koodissa määritelty tietokenttä, joka sisältää jonkin ohjelman
käyttämän arvon}
}

\newglossaryentry{staattinen kieli}
{
    name=staattisesti tyypitetty kieli,
    description={(statically typed language) Ohjelmointikieli, joka tietää jokaisen koodissa
esiintyvän muuttujan tietotyypin koko ajan}
}

\newglossaryentry{signatuuri}
{
	name=signatuuri,
	description={(signature) Metodin määritelmä, joka sisältää sekä metodin nimen, että sen
ottamien argumenttien tyypit}
}

\newglossaryentry{noutaja}
{
	name=noutaja,
	description={(getter) Metodi, jonka tehtävänä on tarjota rajoitettu saatavuus johonkin
luokan yksityiseen muuttujan. Ei yleensä ota argumentteja ja palauttaa luokan omistaman toivotun
muuttujan. Nimetään camelCase-tyylillä sijoittamalla muuttujan nimen eteen "get" ("getVariable").
Parantaa enkapsulaatiota ja mahdollistaa esimerkiksi laiskan alustuksen muuttujille}
}

\newglossaryentry{asettaja}
{
	name=asettaja,
	description={(setter) Metodi, jonka tehtävänä on tarjota luokan ulkopuolisille toimijoille
mahdollisuus asettaa arvoja luokan yksityiseen muuttujaan. Ottaa yleensä arvoksi muuttujan toivotun
arvon, eikä palauta mitään. Nimetään camelCase-tyylillä sijoittamalla muuttujan nimen eteen "set"
("setVariable"). Parantaa enkapsulaatiota ja mahdollistaa esimerkiksi uusien arvojen validoinnin
ennen niiden asettamista}
}

\newglossaryentry{rakentaja}
{
	name=rakentaja,
	description={(constructor) Luokkametodi, joka määritellään ja kutsutaan luokan nimellä ja nimensä
mukaisesti rakentaa ja palauttaa uuden luokan instanssin},
	see={luokkametodi,instanssi,java:new}
}

\newglossaryentry{luokkametodi}
{
	name=luokkametodi,
	description={(class method) Metodi, joka on luokan instanssin sijaan sidottu itse luokkaan.
Ei voida käyttää instanssiin sidottuja muuttujia, mutta voi käyttää luokkamuuttujia. Määritellään
yleensä static-avainsanalla ja kutsutaan luokan nimen pistenotaatiolla},
	see={luokkamuuttuja,java:static}
}

\newglossaryentry{luokkamuuttuja}
{
	name=luokkamuuttuja,
	description={(class variable) Muuttuja, joka on luokan instanssin sijaan sidottu itse luokkaan.
Muokattavissa ja tarkasteltavissa jokaisesta luokan instanssista jaetusti. Määritellään yleensä
static-avainsanalla},
	see=java:static
}

\newglossaryentry{taulukko}
{
	name=taulukko,
	description={(array) Tietorakenne, jossa yksittäiset alkiot on säilötty vierekkäisiin
muistipaikkoihin. Vie vähän muistitilaa ja mahdollistaa nopeat haut ja lisäykset jos alkion
indeksi on tiedossa etukäteen. Yleensä taulukon koko on määriteltävä sen luomisen yhteydessä,
tämä on totta myös Javassa. Javassa taulukko luodaan new-avainsanalla lisäämällä taulukon
säilömän tietotyypin perään hakasulkeet, joiden sisään suljetaan taulukon koon määrittelevä
numero. Taulukkotyyppinen muuttuja määritellään lisäämällä muuttujan tietotyypin määritelmän
perään hakasulkeet. Tällöin esimerkiksi int-tyypin kymmenen alkion taulukko, joka on säilötty
"intArray"-muuttujaan määriteltäisiin seuraavasti:\newline{} "int[] intArray = new int[10]"}
}

\newglossaryentry{geneerinen luokka}
{
	name=geneerinen luokka,
	description={(generic class) Luokan ominaisuus, joka määrittää luokan instanssien täyden
tyyppisignatuurin olevan riippuvainen toisesta luokasta. Tämä toinen luokka voidaan määrittää
jokaisen luokan instanssin kohdalla erikseen pienempi kuin- ja suurempi kuin -merkkien väliin
suljetulla notaatiolla. Esimerkiksi kokoelmat tarvitsevat tietoa sisältämiensä olioiden
tyypityksestä. Merkkijonoja sisältävä ArrayList-kokoelma voitaisiin siis luoda seuraavalla 
notaatiolla:\newline{}"new ArrayList<String>()"\newline{}Geneerinen luokka voi olla myös geneerinen useamman luokan suhteen, jolloin luokat erotetaan pilkulla tai riippuvainen toisesta
geneerisestä luokasta, jolloin geneerisyyden notaatiot kirjoitetaan
sisäkkäin:\newline{}"new HashMap<String, ArrayList<Int>>()"},
	see={java:ArrayList,java:HashMap}
}

\newglossaryentry{hajautustaulu}
{
	name=hajautustaulu,
	description={(hash table) Tietorakenne, joka koostuu avain-arvo -pareista. Jokaista avainta
vastaa yksi arvo. Arvot ovat noudettavissa nopeasti avaimen perusteella. Toteutettu Javan
standardikirjaston HashMap-luokassa},
	see={java:HashMap}
}

\newglossaryentry{ajonaikainen kaantaminen}
{
	name=ajonaikainen kaantaminen,
	description={(Just-In-Time compiling, JIT) Ajonaikainen kääntäminen on koodin ajotekniikka,
jossa}
}


\newglossaryentry{kaannetty kieli}
{
	name=kaannetty kieli,
	description={(compiled language) Ohjelmointikieli, jossa ohjelmistot kompiloidaan ennen
ajamista ja ohjelman ajaminen tapahtuu suorittamalla kompilointiprosessin tuottama tiedosto.
Esimerkiksi c ja Rust ovat kompiloituja kieliä}
}


\newglossaryentry{tulkattu kieli}
{
	name=tulkattu kieli,
	description={(intrepeted language) Ohjelmointikieli, joka ajetaan lukemalla kooditiedostot
reaaliajassa. Esimerkiksi Python ja JavaScript ovat tulkattavia kieliä}
}

\newglossaryentry{roskankeruu}
{
	name=roskankeruu,
	description={(garbage collection) Muistinhallinnan tekniikka, jossa ohjelma tietyin väliajoin
ajaa aliohjelman, joka käy läpi kaikki muuttujat muistissa, tarkistaa onko niihin olemassa
viittauksia ja poistaa muuttujat, joilla ei ole enää aktiivista viittausta}
}

\newglossaryentry{ylikuormitus}
{
	name=ylikuormitus,
	description={(overloading) Kahden samannimisen metodin määrittäminen eri argumenteilla.
Argumenttejä voi olla sama määrä, mutta eri tyypeillä, tai eri määrä. Java hakee metodikutsun
yhteydessä automaattisesti kutsuttua signatuuria vastaavan version ylikuormitetusta metodista}
}

\newglossaryentry{pakotettu tyypinmuunnos}
{
	name=pakotettu tyypinmuunnos,
	description={(casting) Operaatio, jossa muuttujan tyyppi yritetään manuaalisesti muuttaa
joksikin toiseksi tyypiksi. Voi epäonnistua. Tapahtuu kirjoittamalla toivottu tietotyyppi
muutettavan muuttujan nimen eteen sulkuihin. Esimerkiksi double-tyypin muuttuja "doubleNumber"
voitaisiin säilöä int-tyypin muuttujaan "intNumber" pakotetun tyyppimuunnoksen avulla seuraavalla
koodinpätkällä:\newline{}"int intNumber = (int)doubleNumber"}
}

\newglossaryentry{lueteltu tyyppi}
{
	name=lueteltu tyyppi,
	description={(enumerated type) Ohjelmoijan määrittelemä tyyppi, joka koostuu rajatusta
arvojoukosta. Jokaisella arvojoukon arvolla on nimi, jolla siihen viitataan koodissa. Ohjelma
käsittelee kuitenkin arvoja kokonaislukuina, nopeuttaen ajamista. Javassa lueteltu tyyppi luodaan
enum-avainsanalla},
	see={java:enum}
}

\newglossaryentry{serialisaatio}
{
	name=serialisaatio,
	description={(serialization) Erityisesti olio-ohjelmoinnissa yleinen ohjelmointikielten
ominaisuus, joka mahdollistaa olioiden esittämisen tekstimuotoisena datana. Tällä tavalla
esitettyjä olioita on helpompi käsitellä esimerkiksi tiedostojenhallinnan yhteydessä tai
verkkoliikenteessä}
}

\newglossaryentry{periytyminen}
{
	name=periytyminen,
	description={(inheritance) Olio-ohjelmoinnin konsepti, jossa luokat voivat periä toisen luokan
ominaisuudet. Luokkaa, joka perii toisen luokan kutsutaan lapsiluokaksi ja luokkaa, jolla on
lapsiluokka kantaluokaksi. Lapsiluokka sisältää automaattisesti kaikki kantaluokan kentät, eli
muuttujat ja metodit, joiden näkyvyysmääre on protected tai laajempi. Kenttiä voidaan tarvittaessa
ylikirjoittaa},
	see={lapsiluokka,kantaluokka,näkyvyysmääre,java:protected,java:extends}
}

\newglossaryentry{lapsiluokka}
{
	name=lapsiluokka,
	description={(child class) Luokka, joka perii toisen luokan},
	see={periytyminen,kantaluokka}
}

\newglossaryentry{kantaluokka}
{
	name=kantaluokka,
	description={(parent class) Luokka, josta on periytetty vähintään yksi toinen luokka},
	see={periytyminen,lapsiluokka}
}

\newglossaryentry{UML}
{
	name=UML,
	description={(Unified Modelin Language) Mallinnustekniikka ohjelmistojen rakenteen esittelyyn.
UML-standardi sisältää monia erilaisia kaavioita, mutta olio-ohjelmoinnin kannalta niistä tärkein
on luokkakaavio. Muita tärkeitä kaavioita ovat oliokaavio, tilakaavio ja sekvenssikaavio},
	see=luokkakaavio
}

\newglossaryentry{luokkakaavio}
{	
	name=luokkakaavio,
	description={(class diagram) UML-standardin kaavio, joka kuvaa luokkien välisiä suhteita
ohjelmistossa. Koostuu luokkia esittävistä yleensä neliön muotosista soluista. Solut sisältävät
luokkien tietokentät ja metodit. Luokkia voidaan yhdistää toisiinsa erilaisilla nuolilla
osoittamaan niiden välisiä suhteita kuten periytymistä, kompositiota ja aggregaatiota},
	see={periytyminen,kompositio,aggregaatio}
}

\newglossaryentry{kompositio}
{
	name=kompositio,
	description={(composition) Luokkien suhteita kuvaava termi. Kompositiossa luokka B on sidottu
luokan A elinikään niin, että jokainen A:n instanssi sisältää vakiomäärän viittauksia B:n
instansseihin (yleensä 1). Kyseiset B:n instanssit luodaan yhdessä ne omistavan A:n instanssin
kanssa ja ne tuhotaan yhdessä ne omistavan A:n instanssin kanssa. Kompositiossa siis luokka B on
riippuvainen luokasta A olemassaolonsa ajan}
}

\newglossaryentry{aggregaatio}
{
	name=aggregaatio,
	description={(aggregation) Luokkien suhteita kuvaava termi. Aggregaatiossa luokka A omistaa
viittauksen luokan B instanssiin, mutta tämän instanssin luonti ja tuhoutuminen ei ole sidottu
luokan A elinkaareen. Esimerkiksi aktiivisia pelaajia säilövän peli-luokan suhde pelaaja-luokkaan
on aggregaatio, mikäli pelaajat säilötään peli-luokan sisällä listassa, johon jo olemassa oleva
pelaaja lisätään tämän liittyessä peliin ja josta pelaaja poistetaan tämän poistuessa pelistä}
}

\newglossaryentry{korvaaminen}
{
	name=korvaaminen,
	description={(overriding) Lapsiluokan perimän metodin toiminnallisuuden uudelleen määrittely.
Tapahtuu määrittelemällä lapsiluokassa metodin, jonka signatuuri on identtinen kantaluokan
korvattavan metodin kanssa. Tämän metodin toiminnallisuus ajetaan kantaluokan metodin sijaan},
	see={periytyminen,signatuuri,kantaluokka,lapsiluokka}
}

\newglossaryentry{SOLID-periaatteet}
{
	name=SOLID-periaatteet,
	description={(SOLID-principles) Viisi oliopohjaisen ohjelmoinnin käyttöön kehitettyä
periaatetta, joiden tavoitteena on auttaa laadukkaan, siistin, helposti laajennettavan ja helposti
luettavan oliopohjaisen ohjelmiston tuottamisessa. Periaatteet ovat "yhden vastuun periaate"
(Single Responsibility Principle - SRP), "avoin/suljettu-periaate" (Open/Closed Principle - OCP),
"Liskovin korvaavuusperiaate" (Liskov Substitution Principle - LSP), "rajapintojen erottelu
-periaate" (Interface Segregation Principle - ISP) ja "käänteisten riippuvuuksien periaate"
(Dependency Injection Principle - DIP)},
	see={yhden vastuun periaate,  avoin/suljettu-periaate, Liskovin korvaavuusperiaate, 
	rajapintojen erottelu -periaate, kaanteisten riippuvuuksein periaate}
}

\newglossaryentry{yhden vastuun periaate}
{
	name=yhden vastuun periaate,
	description={(Single Responsibility Principle - SRP) SOLID-periaatteisiin kuuluva sääntö, joka
käsittelee koodin jakamista luokkiin. Periaatteen mukaan jokaisella luokalla pitäisi olla yksi ja
vain yksi vastuu. Esimerkiksi String-luokan vastuu on vain merkkijonotyyppisen datan säilöminen ja
sen vastuisiin ei kuulu esimerkiksi datan esittäminen},
	see={SOLID-periaatteet}
}

\newglossaryentry{avoin/suljettu-periaate}
{
	name=avoin/suljettu-periaate,
	description={(Open/Closed Principle - OCP) SOLID-periaatteisiin kuuluva sääntö, jonka mukaan
kirjoitetun koodin on oltava suljettu muutoksilta, mutta avoin laajennukselle. Ohjelmiston osien
on säännön mukaan tapa laajentaa toimintaansa muokkaamatta olemassaolevaa toiminnan määrittelevää
koodia. Esimerkiksi mikäli ohjelmisto käyttää switch-rakennetta, jonka sisällä suoritetaan
logiikkaa saapuvan olion tyypin mukaan, ei rakenne ole laajennettavissa toiminnallisuudeltaan,
ilman että rakenteen koodia muokataan. Sen sijaan rakenne, joka on ohjelmoitu vastaanottamaan
olio ja kutsumaan oliossa itsessään määriteltyä metodia on laajennettavissa toiminnallisuudeltaan
muokkaamatta lähdekoodia. Tämä tapahtuu tarjoamalla rakenteelle erilainen olio, joka toteuttaa
rajapinnan, jossa rakenteen kutsuma metodi on määritelty},
	see={SOLID-periaatteet}
}

\newglossaryentry{Liskovin korvaavuusperiaate}
{
	name=Liskovin korvaavuusperiaate,
	description={(Liskov Substitution Principle - LSP) SOLID-periaatteisiin kuuluva sääntö, joka
käsittelee kanta- ja lapsiluokkien suhdetta. Liskovin korvaavuussäännön mukaan, lapsiluokan
instanssia pitää pystyä käsittelemään kuin kantaluokan instanssia, eli sen on paljastettava
vähintään samat metodit ja muuttuvat kuin kantaluokan instanssien. Korvaavuus ei kulje molempiin
suuntiin, vaan lapsiluokan instanssit saavat säännön mukaan sisältää metodeja ja muuttujia, joita
kantaluokan instansseista ei löydy},
	see={SOLID-periaatteet,kantaluokka,lapsiluokka,periytyminen}
}

\newglossaryentry{rajapintojen erottelu -periaate}
{
	name=rajapintojen erottelu -periaate,
	description={(Interface Segregation Principle - ISP) SOLID-periaatteisiin kuuluva sääntö, joka
käsittelee rajapintojen toiminnallisuuksien rajaamista. Periaatteen mukaan rajapinnan
implementoivan luokan ei pitäisi joutua määrittelemään metodeja, joita se ei tarvitse. Käytännössä
siis jokainen rajapinta pitäisi suunnitella tarpeeksi pieneksi, jotta kaikki sen implementoivat
luokat käyttävät kaikkia sen metodeja}
	see={SOLID-periaatteet,rajapinta,java:interface}
}

\newglossaryentry{kaanteisten riippuvuuksien periaate}
{
	name=kaanteisten riippuvuuksien periaate,
	description={(Dependency Inversion Principle -DIP) SOLID-periaatteisiin kuuluva sääntö, joka
käsittelee ohjelmiston riippuvuuksien suuntaa. Periaatteen mukaan ohjelmiston korkean tason
kompinenttien ei pitäisi olla riippuvaisia matalan tason komponenteista vaan molempien olisi
oltava riippuvainen jaetusta abstraktiosta (toteutettu rajapintaluokkana). Samaan tapaan
rajapintaluokkien ei tulisi olla riippuvaisia konkreettisista luokista, vaan konkreettisten 
luokkien tulisi aina olla riippuvaisia rajapintaluokista},
	see={SOLID-periaatteet,rajapinta,java:import}
}

\newglossaryentry{abstrakti luokka}
{
	name=abstrakti luokka,
	description={(abstract class) Luokka, joka on määritelty abstract-avainsanalla ja jolle on
määritelty vähintään yksi abstrakti metodi. Abstraktista luokasta ei voi luoda instanssia, vaan
luokkaa voi käyttää vain kantaluokkana luokasta luoduille konkreettisille luokille. Näiden
konkreettisten luokkien on implementoitava kaikki luokan abstraktit metodit},
	see={kantaluokka,periytyminen,java:abstract,abstrakti metodi,konkreettinen luokka}
}

\newglossaryentry{abstrakti metodi}
{
	name=abstrakti metodi,
	description={(abstract method) Metodi, joka on määritelty abstract-avainsanalla ja joka
sijaitsee abstraktissa luokassa. Abstraktin metodin määritelmä ei sisällä metodin runkoa,
ainoastaan signatuurin. Luokan konkreettisten lapsiluokkien on implementoitava kyseinen metodi},
	see={abstrakti luokka,signatuuri,java:abstract}
}

\newglossaryentry{konkreettinen luokka}
{
	name=konkreettinen luokka,
	description={(concrete class) Luokka, joka on abstraktin luokan lapsiluokka eikä ole itse
abstrakti luokka. Luokan on implementoitava kaikki kantaluokan abstraktit metodit},
	see={abstrakti luokka,abstrakti metodi}
}

\newglossaryentry{toteuttaa}
{
	name=toteuttaa,
	description={(implement) Rajapinnan toteuttaminen tarkoittaa, että jokin luokka tai metodi
toimii kuten jokin rajapintamääritelmä sanoo sen toimivan. Javassa puhutaan rajapintaluokan
toteuttamisesta käytettäessä implements avainsanaa luokan määritelmässä. Tällöin rajapinnan
toteuttavan luokan voi olettaa toimivan kuten rajapintaluokka määritää, eli luokkaa voidaan
käsitellä kuten toteutetun rajapintaluokan instanssia},
	see={rajapinta,luokka,java:implements}
}

\newglossaryentry{moniperinta}
{
	name=moniperinta,
	description={(multiple inheritance) Periytymisen tekniikka, jossa yhdellä lapsiluokalla on
useampi kantaluokka. Suora moniperintä ei ole tuettua javassa, mutta useamman rajapintasopimuksen
implementoiti yhteen luokkaan onnistuu rajapintaluokilla},
	see={periytyminen,lapsiluokka,kantaluokka,rajapinta,implements,interface}
}
\newglossaryentry{java:interface}
{
    type=java,
    name=interface,
    description={Luo abstraktin rajapinnan, jonka luokka voi implementoida. Näin luotu
rajapintapohja ei voi sisältää datakenttiä eikä sitä voi käyttää olion muottina. Sen sijaan luokka
voi implementoida rajapinnan jolloin luokka lupaa muulle ohjelmistolle tarjoavansa rajapinnan
määrittelemät funktiot}
}

\newglossaryentry{java:bool}
{
    type=java,
    name=bool,
    description={Primitiivinen tietotyyppi, joka sisältää totuusarvon (true tai false)},
    see={primitiivinen tietotyyppi}
}

\newglossaryentry{java:byte}
{
    type=java,
    name=byte,
    description={Primitiivinen tietotyyppi, joka sisältää tavun kokoisen merkillisen
kokonaisluvun},
    see={primitiivinen tietotyyppi}
}

\newglossaryentry{java:short}
{
    type=java,
    name=short,
    description={Primitiivinen tietotyyppi, joka sisältää kahden tavun kokoisen merkillisen
kokonaisluvun},
    see={primitiivinen tietotyyppi}
}

\newglossaryentry{java:int}
{
    type=java,
    name=int,
    description={Primitiivinen tietotyyppi, joka sisältää 32 bitin kokoisen merkillisen
kokonaisluvun},
    see={primitiivinen tietotyyppi}
}

\newglossaryentry{java:long}
{
    type=java,
    name=long,
    description={Primitiivinen tietotyyppi, joka sisältää 64 bitin kokoisen merkillisen
kokonaisluvun},
    see={primitiivinen tietotyyppi}
}

\newglossaryentry{java:float}
{
    type=java,
    name=float,
    description={Primitiivinen tietotyyppi, joka sisältää 32 bitin kokoisen liukumaesitetyn
desimaaliluvun},
    see={primitiivinen tietotyyppi}
}

\newglossaryentry{java:double}
{
    type=java,
    name=double,
    description={Primitiivinen tietotyyppi, joka sisältää 64 bitin kokoisen liukumaesitetyn
desimaaliluvun},
    see={primitiivinen tietotyyppi}
}

\newglossaryentry{java:char}
{
    type=java,
    name=char,
    description={Primitiivinen tietotyyppi, joka sisältää kahden tavun kokoisen unicode-koodatun
merkin esitettynä merkittömänä kokonaislukuna},
    see={primitiivinen tietotyyppi}
}

\newglossaryentry{java:String}
{
    type=java,
    name=String,
    description={Javan standardikirjaston merkkijonoimplementaatioluokka. Suositellaan
käytettäväksi merkkijonojen säilömiseen koodissa. Pystyy säilömään dynaamisen merkkijonon, jonka
alustus- tai maksimikokoa ei tarvitse määrittää erikseen. Lisäksi sisältää lukuisia merkkijonon
käsittelyä ja muokkausta helpottavia metodeja}
}

\newglossaryentry{java:private}
{
	type=java,
	name=private,
	description={Näkyvyysmääre, joka määrittää ominaisuuden olevan käytettävissä vain ominaisuuden
omistaman luokan sisällä.}
}

\newglossaryentry{java:protected}
{
	type=java,
	name=protected,
	description={Näkyvyysmääre, joka määrittää ominaisuuden olevan käytettävissä ominaisuuden
omistavan luokan sisältävässä packagessa ja kaikissa ominaisuuden omistavan luokan perivissä
luokissa.}
}

\newglossaryentry{java:public}
{
	type=java,
	name=public,
	description={Näkyvyysmääre, joka määrittää ominaisuuden olevan käytettävissä kaikkialla
ohjelmistossa.}
}

%%%%%%%%%%%%%%%%%%%%%%%%%%%%%%%%%%%%%%%%%%%%%%%%%%%%%%%%%%%%%%%%%%%%%%%%%%%%%%%%%%%%%%%%%%%%%%%%%%
% % % % % % % % % % % % % % % % % % % % % % % % % % % % % % % % % % % % % % % % % % % % % % % % %
%%%%%%%%%%%%%%%%%%%%%%%%%%%%%%%%%%%%%%%%%%%%%%%%%%%%%%%%%%%%%%%%%%%%%%%%%%%%%%%%%%%%%%%%%%%%%%%%%%

\begin{document}

% Front matter
\frontmatter

\maketitle

\newpage
\begin{fullwidth}
~\vfill
\thispagestyle{empty}
\setlength{\parindent}{0pt}
\setlength{\parskip}{\baselineskip}
Copyright \copyright\ \the\year\ \thanklessauthor

\par\smallcaps{Published by \thanklesspublisher}

\par\smallcaps{tufte-latex.googlecode.com}

\par Licensed under the Apache License, Version 2.0 (the ``License''); you may not
use this file except in compliance with the License. You may obtain a copy
of the License at \url{http://www.apache.org/licenses/LICENSE-2.0}. Unless
required by applicable law or agreed to in writing, software distributed
under the License is distributed on an \smallcaps{``AS IS'' BASIS, WITHOUT
WARRANTIES OR CONDITIONS OF ANY KIND}, either express or implied. See the
License for the specific language governing permissions and limitations
under the License.\index{license}

\par\textit{First printing, \monthyear}
\end{fullwidth}

\tableofcontents

%%%%%%%%%%%%%%%%%%%%%%%%%%%%%%%%%%%%%%%%%%%%%%%%%%%%%%%%%%%%%%%%%%%%%%%%%%%%%%%%%%%%%%%%%%%%%%%%%%
% % % % % % % % % % % % % % % % % % % % % % % % % % % % % % % % % % % % % % % % % % % % % % % % %
%%%%%%%%%%%%%%%%%%%%%%%%%%%%%%%%%%%%%%%%%%%%%%%%%%%%%%%%%%%%%%%%%%%%%%%%%%%%%%%%%%%%%%%%%%%%%%%%%%

\mainmatter

%%%%%%%%%%%%%%%%%%%%%%%%%%%%%%%%%%%%%%%%%%%%%%%%%%%%%%%%%%%%%%%%%%%%%%%%%%%%%%%%%%%%%%%%%%%%%%%%%%
% % % % % % % % % % % % % % % % % % % % % % % % % % % % % % % % % % % % % % % % % % % % % % % % %
%%%%%%%%%%%%%%%%%%%%%%%%%%%%%%%%%%%%%%%%%%%%%%%%%%%%%%%%%%%%%%%%%%%%%%%%%%%%%%%%%%%%%%%%%%%%%%%%%%

\chapter{Oppaasta ja sen käytöstä}
\label{oppaasta}

%%%%%%%%%%%%%%%%%%%%%%%%%%%%%%%%%%%%%%%%%%%%%%%%%%%%%%%%%%%%%%%%%%%%%%%%%%%%%%%%%%%%%%%%%%%%%%%%%%

\section{Esipuhe}

\newthought{Tämä opas} on kirjoitettu osaksi \lutin olio-ohjelmoinnin perusteiden kurssin
kurssimateriaaleja. Opas käy läpi koko kurssin  teoriasisällön lukuunottamatta Android-API:n
käyttöä. Opas on tarkoitettu käytettäväksi osana kurssikokonaisuutta, mutta se on suunniteltu
niin, että se tarjoaa laadukkaan perehdytyksen oliopohjaiseen ajatteluun, suunnitteluun ja
ohjelmointiin myös itsenäisenä kokonaisuutena.

\newthought{Oppaan tavoitteena} on tarjota opiskelijoille helppokäyttöinen,
laadukas ja suomenkielinen resurssi kurssin käsittelemien aihealueiden
opiskeluun. Opas olettaa käyttäjän hallitsevan proseduraalisen ohjelmoinnin
perusteet, jotka käydään läpi esimerkiksi \lutin "Ohjelmoinnin Perusteet" 
-kurssilla.

%%%%%%%%%%%%%%%%%%%%%%%%%%%%%%%%%%%%%%%%%%%%%%%%%%%%%%%%%%%%%%%%%%%%%%%%%%%%%%%%%%%%%%%%%%%%%%%%%%

\section{Oppaan rakenne}

\newthought{Opas on jaettu} kahteentoista sisältölukuun ja kahteentoista bonuslukuun. Sisältöluvut
käsittelevät kurssilla käsiteltäviä sisältöjä ja sisältävät kaiken kurssin sisältöön kuuluvan
materiaalin. Sisältöluvut on numeroitu välillä 0-11 niin, että luvun numero vastaa viikkoa, jolla
luvun sisältöä käsitellään kurssilla ja luku 0 käsittelee itse opasta ja sen käyttöä. Näin opasta
on helppo seurata kurssin edetessä.

\newthought{Jokaista sisältölukua} vastaa oppaan toisella puoliskolla yksi bonusluku. Bonusluvut
tarjoavat laajempia ja syvempiä katsauksia Javaan ja oliopohjaiseen ohjelmointiin. Ne eivät aina
syvennä suoraan vastaavan sisältöluvun materiaaliin liittyviä aiheita, mutta myös niiden järjestys
on suunniteltu niin, että käyttäjä voi opiskella yhden luvun viikossa.

\newthought{Bonusluvut on tarkoitettu} hyödyllisiksi ja hauskoiksi lisälukemistoiksi. Opas on
suunniteltu niin, että bonusluvut eivät sisällä kurssin oppimistavoitteisiin luokiteltua sisältöä.
Niiden lukeminen on siis täysin vapaaehtoista.

%%%%%%%%%%%%%%%%%%%%%%%%%%%%%%%%%%%%%%%%%%%%%%%%%%%%%%%%%%%%%%%%%%%%%%%%%%%%%%%%%%%%%%%%%%%%%%%%%%

\section{Termistö ja kieli oppaassa}

\newthought{Oppaan pääasiallisena} kielenä koodiesimerkkejä lukuunottamatta toimii suomi muun
kurssimateriaalin kielivalinnan mukaan. Oppaan käyttämä termistö on tämän vuoksi käännetty Javan
avainsanoja lukuunottamatta suomeksi. Olisi kuitenkaan turha kiistää englannin ehdotonta ylivaltaa
tietojenkäsittelytieteen pääkielenä. Tämän vuoksi oppaan sanastosta löytyy jokaisen termin
selityksen lisäksi termin englanninkielinen vastine.

\newthought{Opas pyrkii} korostamaan tärkeää termistöä sanaston opettelun helpottamiseksi.
Jokainen tärkeä termi on kerätty oppaan sanastoon lyhyen selityksen ja termin englanninkielisen
vastineen kera. Sanastosta löytyvät termit on myös korostettu ensimmäisessä esiintymisessään.
Sanastotermeillä tämä korostus näkyy kursivoituna kirjoitusasuna ja termin perässä esiintyvänä
sulkuihin suljettuna englanninkielisenä vastineena termille. Vastine on niin ikään kursivoitu.
Koko korostettu kirjoitusasu termille näyttää siis tältä: \newengnogls{esimerkki}{example}

\newthought{Opaan teoriateksti} sisältää sanastosta löytyvien konseptuaalisesti tärkeiden termien
lisäksi myös Javan avainsanoja. Nämä on erotettu muusta korostetusta termistöstä alleviivauksilla
ja kääntämättömyydellä. Ensimmäisellä esiintymiskerrallaan Javan avainsanat ovat lisäksi
kursivoituja. Ensimmäistä kertaa esiintyvä Javan avainsana näyttää tältä: \newjavanogls{example}
ja uudestaan mainittu tältä: \javanogls{example}.

%%%%%%%%%%%%%%%%%%%%%%%%%%%%%%%%%%%%%%%%%%%%%%%%%%%%%%%%%%%%%%%%%%%%%%%%%%%%%%%%%%%%%%%%%%%%%%%%%%

\section{Koodiesimerkit oppaassa}

\newthought{Opas tarjoaa} kattavat koodiesimerkit kaikista siinä käsiteltävistä aihealueista.
Koodiesimerkit ovat lyhyitä, toimivia, koodinpätkiä ja ne sijaitsevat heti esitellyn konseptin
yhteydessä. Kaikki oppaan koodiesimerkit on koottu yhteen *tänne*. %TODO lisää koodien sijainti!

\newthought{Koodiesimerkit on kirjoitettu} Javan virallisen tyylioppaan suosittelemien
ohjelinjojen mukaisesti. Kaikki muuttujat, luokat ja metodit on nimetty englanniksi, mutta koodi
on kommentoitu suomeksi. Kommentointi ei noudata hyvää ohjelmointityyliä vaan kommentit ovat
koodin opetusmaisen luonteen vuoksi monisanaisempaa ja yleisempää kuin suosituksissa.

%%%%%%%%%%%%%%%%%%%%%%%%%%%%%%%%%%%%%%%%%%%%%%%%%%%%%%%%%%%%%%%%%%%%%%%%%%%%%%%%%%%%%%%%%%%%%%%%%%
% % % % % % % % % % % % % % % % % % % % % % % % % % % % % % % % % % % % % % % % % % % % % % % % %
%%%%%%%%%%%%%%%%%%%%%%%%%%%%%%%%%%%%%%%%%%%%%%%%%%%%%%%%%%%%%%%%%%%%%%%%%%%%%%%%%%%%%%%%%%%%%%%%%%

\chapter{Oliopohjainen ajattelu}
\label{olioista}

%%%%%%%%%%%%%%%%%%%%%%%%%%%%%%%%%%%%%%%%%%%%%%%%%%%%%%%%%%%%%%%%%%%%%%%%%%%%%%%%%%%%%%%%%%%%%%%%%%

\section{Abstraktio laadukkaan koodin pohjana}

\newthought{Oliopohjainen ohjelmointi} on ohjelmointiparadigma, joka on kehitetty vastaamaan
ohjelmistotuotannon peruskysymykseen: kuinka kirjoittaa ymmärrettävää ja ylläpidettävää koodia
helposti?

\newthought{Yksi helpoimmista} ja yleisimmistä tavoista selkeyttää koodikantaa on
\neweng{abstraktio}{abstraction}: yleisten toimintojen ja kutsusarjojen eristäminen ennalta
määriteltyihin koodipaloihin. Nämä palat tunnetaan nimellä \neweng{funktio}{function}. Tällä
tavalla jaettuja kutsusarjoja voidaan uusiokäyttää ja ohjelman muokkaaminen helpottuu, kun
kutsusarjan päivittäminen tapahtuu vain yhdessä keskitetyssä paikassa.

\newthought{Käskysarjat ovat} kuitenkin vain puolikas toimivasta ohjelmistosta. Toinen ja aivan
yhtä tärkeä puoli on data, jota käskysarjoilla käsitellään. Myös tämän datan abstraktio on
mahdollista proseduraalisen ohjelmoinnin puitteissa muokattavilla tietorakenteilla. Tällaisestä
hyviä esimerkkejä ovat esimerkiksi Pythonin class-avainsanalla määritellyt luokat, jos niitä
käytetään vain datan ryhmittämiseen tai C-kielen \neweng{tietue}{struct}.

\newthought{Nämä abstraktiotyypit} ovat kuitenkin erotettuja toisitaan, eivätkä huomioi
käskysarjojen ja datan yhteyttä. Lähes poikkeuksetta tiettyjä funktioita kutsutaan koodissa
monessa paikkassa parametreinään toistuvasti sama tietorakenne. Eikö siis olisi järkevää yhdistää
datan ja sitä käsittelevien funktioiden sijainti koodissa?

%%%%%%%%%%%%%%%%%%%%%%%%%%%%%%%%%%%%%%%%%%%%%%%%%%%%%%%%%%%%%%%%%%%%%%%%%%%%%%%%%%%%%%%%%%%%%%%%%%

\section{Oliopohjaisen ajattelun perusteet}

\newthought{Olio-ohjelmoinnissa ohjelmisto} koostuu kokonaan keskenään kommunikoivista yksiköistä.
Näille yksiköille vakiintunut kutsumanimi on \neweng{olio}{object}. Oliot koostuvat datasta ja
sitä manipuloivista käskysarjoista, jotka tunnetaan nimellä \neweng{metodi}{method}. Se on
oliopohjaisen ohjelmoinnin nimi olioon sidotulle funktiolle, termiä \gls{funktio} ei juuri
käytetä.

\newthought{Ohjelmisto luo} kaikki käyttämänsä oliot ohjelmoijan tekemien muottien pohjalta.
Tälläinen muotti, eli \neweng{luokka}{class} voi luoda itsestään yleensä rajattoman määrän olioita
ajon aikana. Luokan ja olion ero on yksi oliopohjaisen ohjelmoinnin tärkeimmistä käsitteistä ja
sen ymmärtäminen alusta lähtien on kriittistä oliopohjaisen ohjelmoinnin opiskelussa.

\newthought{Jokainen olio} on jonkin luokan \neweng{instanssi}{instance}. Saman luokan eri
instanssit omistavat identtiset metodit, mutta instansseihin säilötty data on uniikki jokaiselle
instansille ja instanssin metodit käsittelevät tätä uniikkia dataa. Luokka määrittää nämä metodit
ja kertoo minkä tyyppiset datakentät jokaiselta luokan instanssilta löytyvät. Tästä lähtien
opas käyttää termiä "luokka" viitatessaan ohjelmoijan määrittämään olio muottiin eli puhuttaessa
suunnitteluperiaatteista ja koodin kirjoittamisesta ja termiä "olio" vain puhuessaan selkeästi
koodissa sijaitsevasta luokan ilmentymästä.

\newthought{Tälläinen jaottelu} järjestää koodia proseduraalista koodia luonnollisempiin osiin.
Tästä syystä oliopohjaista ohjelmointia esitellessä mainitaan usein sen kyky luoda koodin sisällä
järjestelmiä, jotka muistuttavat todellisen maailman rakenteita. Vaikka tämä pitää paikkansa on
hyödyllistä ymmärtää alusta lähtien, ettei oliopohjaisen ohjelmoinnin tavoitteena ole luoda yksi
yhteen oikean maailman kanssa samoin toimivaa järjestelmää. Sen sijaan oikea oliopohjaisen
ohjelmoinnin tavoite on luoda järjestelmä, joka koostuu pienistä tarkasti toisistaan erotelluista
osista, jotka kommunikoivat keskenään vain selkeästi määriteltyjä rajapintoja pitkin.

\newthought{Käytännössä laadukasta} oliopohjaista ohjelmistoa voidaan täten verrata esimerkiksi
moderniin modulaarisesti suunniteltuun autoon. Siinä missä auton moottori, pyörät tai verhoilu
ovat muokattavissa asiakkaan toiveiden ja budjetin mukaan, on laadukkaan oliopohjaisen ohjelmiston
osien, kuten vaikka autorisaatiomoduulin tai tietokantayhteyden laajentaminen ja muokkaaminen
mahdollista koskematta muuhun ohjelmistoon.

\newthought{Tämä kuulostaa} kunnianhimoiselta ja monimutkaiselta tavoitteelta, joten on tärkeää
edetä sopivan pienissä paloissa. Samoin kuin laadukas oliopohjainen ohjelmisto koostuu pienistä
paloista, kasataan ohjelmiston tekemiseen vaadittu tieto pieni pala kerrallaan. Tässä kappaleessa
esiteltyjä termejä ja käsitteitä tullaan syventämään oppaan tulevissa kappaleissa yksitellen,
joten vaikka tiedon määrä voi tuntua nyt kohtuuttomalta ei kannata huolestua liikaa.

%%%%%%%%%%%%%%%%%%%%%%%%%%%%%%%%%%%%%%%%%%%%%%%%%%%%%%%%%%%%%%%%%%%%%%%%%%%%%%%%%%%%%%%%%%%%%%%%%%

\section{Koodin laadun mittaamisesta}

\newthought{Jotta ohjelmistojen} modulaarisuus saadaan toivotulle tasolle on tärkeää miettiä mitä
metodeja tai dataa luokka paljastaa muulle ohjelmistolle. Oliopohjaiset kielet sisältävätkin
yleensä jonkin keinon rajoittaa datakenttien ja metodien näkyvyyttä luokan ulkopuolelle. Tämä
näkyvyyden rajaaminen ja tarkkojen vastuiden määrittäminen tunnetaan nimellä
\neweng{enkapsulaatio}{encapsulation}.

\newthought{Luokan paljastamat} datakentät ja metodit kertovat ulkopuoliselle ohjelmistolle, mitä
luokasta luodut oliot voivat tehdä ja mitä niiltä voi odottaa. Tämä muodostaa sopimuksen luokan ja
muiden luokan paljastamia metodeja tai datakenttiä käyttävien luokkien välillä. Tätä sopimusta
kutsutaan nimellä \neweng{rajapinta}{interface}. Rajapinnan abstraktia käsitettä ei pidä sekoittaa
Javan samankaltaiseen "interface"-avainsanaan, joka esitellään myöhemmin.

\newthought{Yksi olio-ohjelmoinnin nyrkkisäännöistä} on pitää huolta, että jokainen luokka
suorittaa yhden ja vain yhden tehtävän. Tämä tarkoittaa luokan rajapinnan pitämistä minimaalisena,
eli korkeaa enkapsulaatiota. Tämän lisäksi ohjelmiston peruslaatua voidaan arvioida käsitteillä
\neweng{koheesio}{cohesion} ja \neweng{pariutuminen}{coupling}.

\newthought{Koheesio voidaan arvioida} luokkakohtaisesti luokan metodien ja datakenttien
yhteistoiminnan perusteella. Mitä tiiviimmin metodit ja datakentät ovat yhteistyössä ja mitä
vähemmän luokassa on metodeja ja datakenttiä, jotka eivät ole vuorovaikutuksissa muun luokan
kanssa, sitä korkeampi koheesio luokalla on. Korkea koheesio kulkee siis pitkälti käsi kädessä
tiukan enkapsulaation kanssa. Mitä vähemmän luokka paljastaa itsestään muulle ohjelmistolle, sitä
tiukemmin sen metodit ja datakentät yleensä komminkoivat keskenään.

\newthought{Pariutuminen tarkoittaa} ohjelmiston luokkien riippuvuutta toisista ohjelmiston
luokista. Siitä puhutaan usein asteikolla \newengnogls{löysä}{loose}-\newengnogls{tiukka}{tight}.
Löysästi pariutineessa ohjelmistossa jokainen luokka on riippuvainen vain muutamasta muusta
luokasta ja riippuvaisuudet muistuttavat usein enemmän ketjua tai puuta kuin verkkoa.

\newthought{Koheesio ja pariutuminen} kulkevat usein käsi kädessä: Korkea koheesio johtaa löysään
pariutumiseen ja matala koheesio tiukkaan pariutumiseen. Enkapsulaatio mukailee yleensä koheesiota
niin että korkeasta koheesiosta seuraa korkea enkapsulaatio ja toisin päin. Kaikki kolme ovat
tärkeitä mittareita puhuttaessa oliopohjaisen ohjelmiston laadusta. Vaikka oppaan tärkein tehtävä
onkin opettaa lukija koodaamaan Javalla, eikä niinkään koodaamaan huipputason koodia, on näiden
käsitteiden olemassaolo ja merkitys hyvä tiedostaa. Tämä auttaa ymmärtääkseen oliopohjaisten
kielten suunnittelua ja etuja suhteessa muihin ohjelmointikieliin ja saattaa myös parantaa
tuotetun koodin laatua huomaamattomasti.

%%%%%%%%%%%%%%%%%%%%%%%%%%%%%%%%%%%%%%%%%%%%%%%%%%%%%%%%%%%%%%%%%%%%%%%%%%%%%%%%%%%%%%%%%%%%%%%%%%
% % % % % % % % % % % % % % % % % % % % % % % % % % % % % % % % % % % % % % % % % % % % % % % % %
%%%%%%%%%%%%%%%%%%%%%%%%%%%%%%%%%%%%%%%%%%%%%%%%%%%%%%%%%%%%%%%%%%%%%%%%%%%%%%%%%%%%%%%%%%%%%%%%%%

\chapter{Javan perusteet}
\label{perusteet}

%%%%%%%%%%%%%%%%%%%%%%%%%%%%%%%%%%%%%%%%%%%%%%%%%%%%%%%%%%%%%%%%%%%%%%%%%%%%%%%%%%%%%%%%%%%%%%%%%%	

\section{Java ohjelmointikielenä}

\newthought{Java} on staattisesti tyypitetty korkean tason oliopohjainen ohjelmointikieli, joka
pyrkii standardisoimaan ohjelmien kirjoittamisen alustasta riippumatta. Java-ohjelmat käännetään
bittikoodiksi, joka ajetaan Java-virtuaalikoneella siten, että tismalleen sama koodi voidaan ajaa
jokaisella alustalla, jolle virtuaalikone voidaan asentaa.

\newthought{Tästä syystä} Java toimii tismalleen samoin kaikilla käyttöjärjestelmillä. Tämä
yhdessä Javan helposti opittavan syntaksin ja oliopohjaiseen suunnitteluun pakottavan luonteen
ohella on tehnyt Javasta yhden käytetyimmistä ohjelmointikielistä teollisessa ohjelmoinnissa.

\newthought{Javan syntaksi} on saanut paljon vaikutteita C-kielestä. Muun muassa kaarisulkeiden
käyttö koodin osien erotteluun ja suurin osa ohjausrakenteiden nimistä on suoraan C-kielestä
kopioitua.

%%%%%%%%%%%%%%%%%%%%%%%%%%%%%%%%%%%%%%%%%%%%%%%%%%%%%%%%%%%%%%%%%%%%%%%%%%%%%%%%%%%%%%%%%%%%%%%%%%

\section{Javan syntaksin alkeet}

\subsection{Kommentointi}

\newthought{Suurin osa} ohjelmointikielistä tukee ainakin yhtä kommentointisyntaksia. Javassa on
kaksi pääasiallista tapaa merkata koodissa oleva teksti kommentiksi. Yhden rivin kommentti voidaan
aloittaa "//"-merkkiparilla, kun taas monen rivin kommentti suljetaan "/*"- ja "*/"-merkkien
väliin. Seuraava esimerkki esittää kommentointisyntaksin vielä tarkemmin. Älä välitä vielä itse
koodista, sen tehtävä on vain selventää miten kommenttien sijoittelu vaikuttaa koodin toimintaan.

\code{week2}{CommentingExample}{Kommentointi Javassa}

\subsection{Primitiiviset tietotyypit ja String}

\newthought{Yksiä tärkeimmistä} koodin osista ovat ehdottomasti \textit{\gls{muuttuja}t}
\eng{variable}. Ne ovat koodissa määriteltyjä tietokenttiä, jotka kykenevät säilömään joko jonkin
luokan instanssin, tai kokonaislukumuodossa esitetyn simppelin datayksikön. Koska Java on 
\neweng{staattinen kieli}{statically typed language}, täytyy siinä jokaisen muuttujan tietotyyppi
määritellä muuttujan määrittelyn yhteydessä. Muuttujan tyyppi voi olla joko jokin käyttäjän
määrittelemä luokka, jokin Javan standardikirjaston luokka, tai jokin \neweng{primitiivinen
tietotyyppi}{primitive data type}.

\newthought{Java sisältää} yhteensä kahdeksan primitiivistiä tietotyyppiä. Nämä määritellään
pienellä alkukirjaimella kirjoitetulla tyypin nimellä. Seuraava taulukko ja koodiesimerkki
sisältävät kaikki primitiivisten tietotyyppien nimet, arvojoukot, oletusarvot ja määrittelytavan
koodissa.

\bigskip
\begin{center}
\footnotesize
\begin{tabular}{lll}
\toprule
Avainsana & Oletusarvo & Arvojoukko \\
\midrule
\newjava{bool} & false & true ja false \\
\addlinespace
\newjava{byte} & 0 & $-2^7$:stä $2^7-1$:een \\
\addlinespace
\newjava{short} & 0 & $-2^{15}$:stä $2^{15}-1$:een \\
\addlinespace
\newjava{int} & 0 & $-2^{31}$:stä $2^{31}-1$:een \\
\addlinespace
\newjava{long} & 0L & $-2^{63}$:stä $2^{63}-1$:een \\
\addlinespace
\newjava{float} & 0.0f & noin $1.4e-45$:stä noin $3.4e+38$:aan \\
\addlinespace
\newjava{double} & 0.0d & noin $4.9e-324d$:stä noin $1.8e+308d$:hen \\
\addlinespace
\newjava{char} & 0 & $0$:sta $2^{16}-1$:een \\
\bottomrule
\end{tabular}
\end{center}

\code{week2}{BasicDataTypes}{Primitiiviset tietotyypit Javassa}

\newthought{Näiden tietotyyppien} lisäksi Javasta löytyy monta valmiiksi määriteltyä luokkaa datan
säilömistä varten. Näistä tärkein kielen opettelua aloitettaessa on ehdottomasti merkkijonoja
säilövä \newjava{String}-luokka. Tämä luokka mahdollistaa dynaamisten merkkijonojen helpon
luomisen ja manipulaation. String-luokka on voimakas työkalu ja sen toimintaan kannattaa
kiinnittää huomiota Javan opiskelun alussa.

\newthought{String-tyyppisen} muuttujan arvo kerrotaan ohjelmalle antamalla String-tyypin
muuttujalle toivottu merkkijono suljettuna normaalien lainausmerkkien (") sisään. Yksittäiset 
hipsut (') eivät toimi, koska niitä käytetään char-tyypin muuttujien määrittämiseen.

\code{week2}{BasicString}{String-tyypin muuttujan määrittäminen}

%%%%%%%%%%%%%%%%%%%%%%%%%%%%%%%%%%%%%%%%%%%%%%%%%%%%%%%%%%%%%%%%%%%%%%%%%%%%%%%%%%%%%%%%%%%%%%%%%%

\section{Toimivan ohjelman perusteet}

\subsection{Oliopohjaisen ohjelman ajaminen: main-metodi}

\newthought{Luokat ovat} Javan ydin ja kaiken koodin rakennuspala. Kaikki koodi Javassa on
suljettava luokan sisään, niin myös ohjelman ajon aloittava koodi. Tästä syystä jokaisessa
Java-ohjelmistossa on yksi erikseen määritelty pääluokka, joka sisältää main-metodin. Tämä
metodi määritellään avainsanoilla \java{public} \java{static} \java{void} ja sen 
\neweng{signatuuri}{signature} on \java{main}(\java{String}[] args). Void-avainsana selitetään
myöhemmin kappaleessa ja static-avainsana kappaleessa *x*. % TODO: lisää kappaleen numero
Toistaiseksi riittää sisällyttää nämä avainsanat annetussa järjestyksessä main-metodin
määritelmään. Pääluokan ei tarvitse olla ainoa main-metodin sisältävä luokka, mutta vain yhden
ennalta määritellyn luokan main-metodi ajetaan ohjelman käynnistyksen yhteydessä.

\subsection{Luokan luomisen syntaksi}

\newthought{Luokka luodaan} Javassa luokat luodaan \newjava{class} avainsanalla. Avainsana
tarkistaa edestään myös näkyvyysmääreen (käsitellään seuraavassa alakappaleessa). Yleisin tapa
luoda luokka on näkyvyysmääreellä \java{public}. Koko määritelmä koostuu näkyvyysmääreestä,
\java{class} -avainsanasta ja luokan nimestä, jota seuraa luokan käyttäytymisen määrittävä koodi
hakasulkeisiin suljettuna. Seuraavassa koodiesimerkissä on kommentoitu auki yksinkertaisen
pääluokan luonti (huomaa pääluokan nimeämisen vapaus) ja sen sisään "Hello world!" tulostavan
main-funktion lisääminen.

\subsection{Tulostaminen Javassa}

\newthought{Tekstin tulostaminen} komentoikkunaan on varmasti tuttua aiemmasta
ohjelmointikokemuksesta. Javassa tulostukseen käytetään yleensä standardikirjaston 
\newjava{System}-luokan out-muuttujassa tallennettuna olevaa PrintStream-luokan instanssia, joka
edustaa oletustulostevirtaa. Jos et ymmärtänyt edellistä lausetta, se ei haittaa. Tärkeintä on
tietää, että Javassa helpoin tapa tulostaa on kutsua "\java{System}.out" - sijainnista sopivaa
tulostusmetodia. Opas on käyttää suurimmaksi osaksi "println"-metodia, joka ottaa merkkijonon ja
tulostaa sen, sekä rivinvaihdon. On huomioitavaa, että println, samoin kuin muutkin
\java{System}.out:sta löytyvät tulostusmetodit ovat todella monipuolisia ja tukevat monenlaisia
tapoja tulostaa toivottua tekstiä. Toistaiseksi nämä ominaisuudet eivät kuitenkaan ole
ajankohtaisia ja opas käyttää "\java{System}.out.println" metodia tulostukseen.

\code{week2}{HelloWorld}{Yksinkertaisen pääluokan ja main-funktion luominen Javassa}

\subsection{Metodin määritteleminen ja kutsuminen}

\newthought{Metodin määritelmä} koostuu Javassa metodin paluuarvon tyypistä, metodin nimestä
ja sitä seuraavasta sulkuihin suljetusta parametrien määritelmästä ja lopulta kaarisulkuihin
suljetusta metodin koodista. Metodin parametrit vaativat pythonista poiketen parametrin nimen
määritelmän lisäksi myös parametrin tyypin määrittämistä. Vastaavasti metodia kutsuttaessa on
argumenttien, jolla metodia kutsutaan oltava metodin yhteydessä määriteltyä tyyppiä.

\newthought{Metodien nimeämisessä} suositellaan Javassa vahvasti niin sanottua camel case-
nimeämistapaa. Tavassa kaikki nimen sanat kirjoitetaan yhteen niin, että nimi aloitetaan
pienellä kirjaimella ja seuraavat sanat aloitetaan isolla alukirjaimella. Täten esimerkiksi
"parse user name"-metodin nimi on muotoa "parseUserName". On myös vahvasti suositeltua käyttää
verbiä metodin nimessä. Tämä helpottaa metodien ja muuttujien erottamista ja pakottaa
myös ajattelemaan koodin laatua: hyvä metodi tekee vain yhden asian, joten jos nimeäminen
simppelissä verbimuodossa ei onnistu, olisi ehkä syytä jakaa metodi useampaan pienempään
metodiin.

\newthought{Metodia kutsutaan} Javassa pistenotaatiolla metodin omistavasta oliosta. Tämä
olio on yleensä tallennettuna muuttujaan jonkin toisen luokan sisällä. Esimerkiksi jos olio
on tallennettu muuttujaan "object" ja omistaa metodin "doStuff(int number)", kutsuttaisiin
metodi notaatiolla "object.doStuff(322)". Jos metodille on määritelty paluuarvon tyyppi, kutsu
palauttaa tämän tyypin muuttujan, jota voidaan käyttää koodissa.

\code{week2}{BasicMethodChild}{Metodin luominen Javassa}
\code{week2}{BasicMethod}{Metodin kutsuminen Javassa}

\subsection{If-lause ja else-lause}

\newthought{Ohjelmointi on} pohjimmiltaan logiikkaa. Ohjelman ajon muokkaaminen olosuheiden
mukaan tekee ohjelmasta älykkään. Yksinkertaisin tapa muokata ohjelman toimintaa on
\newjava{if}-lause ja sen kanssa toimiva \newjava{else}-lause. Lauseet mahdollistavat ennalta
määriteltyjen koodin osien ajamisen annetun lauseen totuusarvon mukaan.

\newthought{If-lause} alkaa if-avainsanalla, jota seuraa suluilla ympäröity lause, joka saa
totuusarvon true tai false. Tätä ehtolausetta seuraa kaarisulkuihin ympäröitynä itse if-lauseen
runko, eli koodi, joka ajetaan jos ehtolause on tosi.

\newthought{Else-lause} on mahdollinen jatko if-lauseelle ja sijaitsee aina if-lauseen
ehdollisesti ajettavan koodiosan jälkeen. Lause alkaa else-avainsanalla. Tämän jälkeen on
mahdollista kirjoittaa uusi if-lause, tai aloittaa suoraan ehdollisen koodin määrittely. If-
else -lauseita voi tarvittaessa ketjuttaa niin monta kuin tarvitaan, joskin pitkät if-else
-ketjut ovat yleensä merkki huonosta arkkitehtuurista ohjelmistossa.

\newthought{Seuraava koodiesimerkki} esittelee if- ja else-lauseiden käytön perusteet. Huomioi,
että normaalisti ehtolauseiden arvoa ei tiedetä ennen ohjelman ajoa, vaan se muuttuu dynaamisesti.

\code{week2}{BasicIfElse}{Esimerkki if- ja else-lauseiden käytöstä}

%%%%%%%%%%%%%%%%%%%%%%%%%%%%%%%%%%%%%%%%%%%%%%%%%%%%%%%%%%%%%%%%%%%%%%%%%%%%%%%%%%%%%%%%%%%%%%%%%%

\section{Javan ominaispiirteitä}

\subsection{Näkyvyysmääreet}

\newthought{Lähes kaikkien} koodin osien, kuten muuttujan, metodin, interface-luokan tai luokan
luominen Javassa tukee kyseisen koodin osan näkyvyyden määrittämistä. Tämä määritelmä on nimeltään 
\neweng{nakyvyysmaare}{access modifier}. Kuten oppaassa aiemmin todettiin, yksi olio-ohjelmoinnin
ydinideoista on \new{enkapsulaatio}, eli koodin osien vastuiden tarkka rajaaminen. Näkyvyysmääreet
ovat tärkein työkalu tämän suhteen. Näkyvyysmääre rajoittaa ohjelmiston osan, kuten luokan tai
luokan ominaisuuden näkyvyyttä muulle ohjelmalle seuraavan taulukon mukaisesti:

\bigskip
\begin{center}
\footnotesize
\begin{tabular}{lll}
\toprule
Avainsana & Näkyvyys & Käyttökohteet \\
\midrule
\newjava{private} & Vain tämä luokka & Luokkien ominaisuudet \\
\addlinespace
Ei avainsanaa (default) & Kaikki luokat tässä paketissa & Luokat ja luokkien ominaisuudet \\
\addlinespace
\newjava{protected} & Kaikki luokat tässä paketissa ja kaikki tämän luokan perivät luokat &
Luokkien ominaisuudet \\
\addlinespace
\newjava{public} & Kaikki luokat kaikkialla & Luokat ja luokkien ominaisuudet \\
\bottomrule
\end{tabular}
\end{center}

\newthought{Private ja protected} ovat siis näkyvyysmääreitä, jotka toimivat vain luokan
ominaisuuksien näkyvyyden rajaamisessa, kun taas public ja oletusnäkyvyys toimivat myös luokkien
näkyvyyden määrittämisessä. Tämä johtuu siitä, että private ja protected ovat suoraan sidottuja
luokkaan, jossa ominaisuus sijaitsee. Protected on lisäksi tiukasti sidoksissa periytymiseen, joka
käsitellään myöhemmin oppaan luvussa *x*, % TODO: lisää luku 
joten sen toiminnallisuuden ymmärtäminen ei ole vielä ajankohtaista.

\newthought{Tähän mennessä} oppaassa on vältetty näkyvyysmääreiden käyttöä, eli niitä on käytetty
vain main-metodien määrittelemisen yhteydessä. Tällöin Java-kääntäjä asettaa näkyvyydeksi
automaattisesti oletusnäkyvyyden (default), jossa luokka, tai sen ominaisuus näkyy kaikkialle
sen määrittelypaketin sisällä, muttei muualle. Tämä ei ole suositeltavaa, vaan jokaisen luokan
ja luokkien jokaisen ominaisuuden näkyvyyden tarve pitäisi arvioida erikseen ja asettaa kullekin
tiukin mahdollinen näkyvyysmääre. Tämä tyyli paitsi takaa mahdollisimman korkean enkapsulaation,
myös tekee ohjelmistosta tehokkaamman, koska vain luokan sisällä näkyvät metodit voidaan optimoida
tehokkaammin ohjelman kääntämisen yhteydessä.

\newthought{Seuraava koodiesimerkki} sisältää pääluokan ja kaksi apuluokkaa ja esittelee näin
\java{public} ja \java{private} avainsanojen toimivuuden. Avainsana \java{protected} on esitelty
koodiesimerkissä periytyvyyden yhteydessä kappaleessa *x*. % TODO: lisää luku

\code{week2}{AccessModifierExampleFirstChild}{Ensimmäinen näkyvyysmääre-esimerkin luokka}
\code{week2}{AccessModifierExampleSecondChild}{Toinen näkyvyysmääre-esimerkin luokka}
\code{week2}{AccessModifierExampleMain}{Näkyvyysmääre-esimerkin pääluokka}

\subsection{Noutajat ja asettajat}

\newthought{Java neuvoo} käyttäjiään alustamaan kaikki muuttujat \java{private} näkyvyysmääreellä.
Tämä tarkoittaa, että koodatessa virallisten suositusten mukaista Javaa ohjelmoija tuottaa vain
luokkia, joiden muuttujat näkyvät ainoastaan kunkin luokan sisällä. Kuitenkin monesti luokan
säilömää dataa tarvitaan luokan ulkopuolella. Tämä ei ole suunnitteluvirhe, vaan yksinkertainen
ohjelmoinnin totuus. Tätä varten Javaan on vakiintunut kaksi tärkeää metodityyppiä:
\neweng{noutaja}{getter} ja \neweng{asettaja}{setter}.

\newthought{Noutaja, eli getter} on funktio, jonka tehtävä on noutaa luokan sisällä yksityiseksi
määritelty muuttuja luokan ulkopuoliseen käyttöön. Noutaja nimetään camelCase-tyylisesti
lisäämällä noudettavan muuttujan nimen eteen "get". Tällöin muuttujan "privateVariable"
noutajafunktion nimi on "getPrivateVariable". Noutajafunktio ei yleensä ota argumentteja ja
palauttaa vain noudettavan muuttujan. Noutajan tehtävä on tarjota rajattu pääsy luokan
säilömään dataan luokan ulkopuolelle. Noutajien käyttö mahdollistaa myös esimerkiksi laiskan
alustamisen muuttujille, joiden alustaminen on raskasta, muttei aina tarpeellista.

\newthought{Asettaja, eli setter} puolestaan on funktio, jonka tehtävä on asettaa luokan
ulkopuolelta saatava arvo luokan sisällä yksityiseksi määriteltyyn muuttujaan. Asettaja nimetään
noutajan tapaan camelCase-tyylisesti lisäämällä muuttujan nimen eteen "set". Täten edellä mainitun
"privateVariable"-muuttujan asettajan nimi on "setPrivateVariable". Asettajafunktio ottaa yleensä
asetettavan muuttujan tyyppiä olevan instanssin eikä palauta mitään. Samoin kuin noutajat,
asettajat rajoittavat luokan ulkopuolista pääsyä luokan säilömään dataan. Lisäksi asettajat
vastaavat usein luokan ulkopuolelta tulevan datan validoinnista ennen sen tallentamista
muuttujaan.

\newthought{Noutajien ja asettajien} käytön perusteet on esitetty seuraavassa koodiesimerkissä.
Huomaa setPositiveNumber-metodissa toteutettava muuttujan validointi, joka ei olisi mahdollista,
jos muuttuja positiveNumber olisi julkinen Javan suositteleman yksityinen muuttuja, noutaja ja
asettaja -mallin sijaan.

\code{week2}{GetterSetterChild}{Ensimmäinen noutaja/asettajaesimerkin luokka}
\code{week2}{GetterSetterMain}{Noutaja/asettajaesimerkin pääluokka}

%%%%%%%%%%%%%%%%%%%%%%%%%%%%%%%%%%%%%%%%%%%%%%%%%%%%%%%%%%%%%%%%%%%%%%%%%%%%%%%%%%%%%%%%%%%%%%%%%%

\section{Syötteen vastaanottamisen perusteet}

\subsection{Käyttäjän syötteen vastaanottaminen Javassa}

\newthought{Komentoriviin perustuvan} ohjelman tuottaminen vaatii paitsi kykyä tulostaa
komentoriville tekstiä, myös kykyä kerätä käyttäjältä syötettä. Tätä varten Javasta löytyy
useampi luokka, joiden avulla voidaan kerätä haluttu syöte juuri ohjelmistolle sopivalla tavalla.
Näistä tärkeimmät ovat \newjava{Scanner}-lukka, \newjava{BufferedReader}- ja
\newjava{InputStreamReader}-luokat. Koska Java, kuten moni muu ohjelmointikieli, käsittelee
konsolisyötettä ja tiedostosyötettä samoin kaikkia näitä luokkia voi käyttää myös tiedostojen
lukemiseen, käyttäjäsyötteen lukemisen lisäksi. Huomioi, että syötteen pyytäminen on matalan
tason tapahtuma, jonka täyden toiminnallisuuden ymmärtäminen ei ole tarpeellista oppaan
käyttämiseksi. Seuraavat kappaleet avaavat koko syötteennoutoprosessin, mutta voit halutessasi
hypätä suoraan seuraavaan koodiesimerkkiin, jos haluat vain opiskella tarvittavan syntaksin.

\newthought{\java{Scanner}-luokka} ottaa sisäänsä jonkin luettavan olion ja mahdollistaa sen
parsimisen. Käyttäjän syöte tulee oletuksena standardikirjaston System-luokan in-muuttujaan,
joten Scanner-luokan rakentaminen System.in-argumentilla on helpoin tapa muodostaa käyttäjältä
syötettä keräävä olio. Tältä oliolta voi pyytää monia eri parsintametodeja, kuten seuraavan rivin
parsimista, tai seuraavan numeron parsimista. Todennäköisesti yleisin parsintametodi on rivin
parsiminen, joka tapahtuu Scanner-olion "nextLine()"-metodilla.

\newthought{\java{Scanner}}, joka on koottu suoraan System.in-streamin
päälle on helppo tapa vastaanotta käyttäjäsyötettä, mutta raa'a stream-olion lukeminen on
kohtalaisen raskas toimenpide. Java suosittelee puskuroimaan useasti luetut syötteet, olivat ne
tiedostoja tai konsolisyötteitä. Tätä varten on olemassa \java{BufferedReader}-luokka. Se ottaa
rakennettaessa puskuroitavan merkkijonovirran. BufferedReader voidaan antaa suoraan argumenttina
Scanner-oliolle luettavaksi syötteeksi.

\newthought{Ikävä kyllä} \java{System}.in ei ole merkkijonovirta, vaan tavuvirta. Sitä ei siis
voida käyttää raakana \java{BufferedReader}-olion rakentamiseen. Tavuvirta voidaan muuttaa
merkkijonovirraksi \java{InputStreamReader}-luokalla. Rakentamalla \java{InputStreamReader}
ja käyttämällä \java{System}.in-virtaa argumenttina, saadaan merkkijonovirta, joka sisältää
käyttäjän syötteen.

\newthought{Syötteen noutaminen} on ikävä kyllä Javassa monimutkainen prosessi. Seuraava
koodiesimerkki sisältää kuitenkin tarvittavan syntaksin toimivan Scanner-luokan instanssin
luomiseen. Kun instanssi on luotu kerran on sen käyttäminen onneksi simppeliä.

\code{week2}{BasicUserInput}{Käyttäjän syötteen noutaminen Javassa}

%%%%%%%%%%%%%%%%%%%%%%%%%%%%%%%%%%%%%%%%%%%%%%%%%%%%%%%%%%%%%%%%%%%%%%%%%%%%%%%%%%%%%%%%%%%%%%%%%%
% % % % % % % % % % % % % % % % % % % % % % % % % % % % % % % % % % % % % % % % % % % % % % % % %
%%%%%%%%%%%%%%%%%%%%%%%%%%%%%%%%%%%%%%%%%%%%%%%%%%%%%%%%%%%%%%%%%%%%%%%%%%%%%%%%%%%%%%%%%%%%%%%%%%

\chapter{Kokoelmarakenteet, toistorakenteet ja lisää oliopohjaista ajattelua}

%%%%%%%%%%%%%%%%%%%%%%%%%%%%%%%%%%%%%%%%%%%%%%%%%%%%%%%%%%%%%%%%%%%%%%%%%%%%%%%%%%%%%%%%%%%%%%%%%%

\section{Kokoelmarakenteet}

\newthought{Ohjelmistot käsittelevät} yleensä huomattavia määriä dataa. Ei ole mielekästä, eikä
yleensä edes mahdollista määritellä kaikille mahdollisille datayksiköille omaa muuttujaansa.
Tätä varten ohjelmointikielistä löytyy kokoelmamaisia rakenteita, jotka ovat nimensä mukaisesti
yhteen muuttujaan tallennettavia datayksikkökokoelmia. Javan standardikirjasto sisältää useita
kokoelmarakenteita erityyppisiin tilanteisiin, mutta opas käy läpi vain yleisimmin käytetyt
kokoelmat.

\subsection{Taulukot}

\newthought{Yksinkertaisin kokoelmarakenteista} on \neweng{taulukko}{array}, joka löytyy
simppeliytensä vuoksi useasta ohjelmointikielestä. Taulukko on tehokas rakenne indeksipohjaisissa
hauissa ja lisäyksissä ja sen ylimääräinen muistijalanjälki on pieni verrattuna muihin
tietorakenteisiin. Oppaassa on jo näkynyt taulukkomuuttujia, sillä \java{main}-funktio ottaa
komentoriviargumentit muodossa String[], eli merkkijonoista koostuvana taulukkona. On huomioitava,
että Java vaatii kaikkien taulukon alkioiden olevan samaa tyyppiä.

\newthought{Taulukko alustetaan} \java{new}-avainsanalla. Avainsanan jälkeen annetaan taulukon
alkoiden tietotyyppi, jonka perään kirjoitetaan välittömästi hakasulkujen sisään taulukon koko.
Muuttuja, joka sisältää taulukon määritellään puolestaan normaalin muuttujan määrittelyn tapaan,
mutta muuttujan tietotyypin perään lisätään tyhjä pari hakasulkuja. Taulukon alkoihin viittaaminen
puolestaan tapahtuu kirjoittamalla viitattavan alkion indeksi hakasulkuihin taulukon sisältävän
muuttujan perään. Viittaus toimii samoin sekä alkion tallennuksessa, että alkion noutamisessa.

\newthought{Seuraava esimerkki} näyttää esimerkin taulukkojen luomisesta ja tallettamisesta
muuttujaan, sekä arvojen tallentamisesta olemassa olevaan taulukkoon ja arvojen noutamisesta
olemassa olevasta taulukosta.

\code{week3}{BasicArrayDataPoint}{Dataluokka käytettäväksi taulukkoesimerkissä}
\code{week3}{BasicArray}{Taulukkojen käyttö Javassa}

\subsection{Listat}

\newthought{Todennäköisesti yleisin} tietorakenne datajoukkojen säilömiseen on lista. Javan
standardikirjasto sisältää useamman listamaisen implementaation, mutta niistä yleisimmin käytetty 
ja monikäyttöisin on \newjava{ArrayList}. Se muistuttaa pythonin listaa toiminnaltaan, mutta
pystyy Javan staattisen tyypityksen vuoksi säilömään vain yhden tyyppisiä muuttujia. Toisaalta on
huomioitava, että mikäli ohjelmassa nousee ikinä tarve säilöä usean tyyppisiä muuttujia samaan
listaan, on ohjelma luultavasti epäoptimaalisesti suunniteltu.

\newthought{\java{ArrayList} tukee} pythonin listan tapaan indeksipohjaista hakua ja lisäystä,
alkoiden hakua kriteeripohjaisesti ja iterointia. Toisin kuin taulukko se on kooltaan dynaaminen
eikä sen kokoa siksi määritellä sen luonnin yhteydessä. Luonti tapahtuu \java{new}-avainsanalla,
mutta koska Javan täytyy tietää paitsi listan tyyppi, myös listan säilömien olioiden tyyppi, on
luomisen yhteydessä määriteltävä tämäkin. ArrayList on siis
\neweng{geneerinen luokka}{generic class}. Geneeristen luokkien luominen ja toimintaperiaate
ei ole tämän luvun aiheena, vaan ne käsitellään oppaan luvussa *x*. % TODO: lisää luku
Tässä kohtaa opasta ArrayListin geneerisyys on huomioitava vain uuden instanssin luonnin
yhteydessä, jolloin listan säilömä tyyppi on suljettava pienempi kuin- ja suurempi kuin -merkkien
väliin. Seuraava esimerkki näyttää ArrayList-instanssin luomisen, alkioiden lisäämisen, hakemisen
ja poistamisen.

\code{week3}{BasicArrayList}{ArrayList-luokan käyttö Javassa}

\subsection{Hajautustaulut}

\newthought{Välillä kokoelman} sisältämää dataa on tarve yhdistää muuhun dataan. Vaikka
olio-ohjelmoinnin hengen mukaisesti datan yhdistäminen tapahtuu luomalla uusi luokka tai
muokkaamalla olemassa olevaa luokkaa ei tämä ole aina mahdollista tai järkevää. Erityisesti
tilanteissa, joissa dataa haetaan jonkin avaimen avulla ei ole järkevää käyttää listaa, jossa on
tämän avaimen omaavia alkioita hitaan ja hankalan hakuoperaation vuoksi. Tämän vuoksi Javassa
on olemassa \newjava{HashMap}-luokka jonka perustana on \neweng{hajautustaulu}{hash table}
-tietorakenne.

\newthought{\java{HashMap}-luokka} säilöö avain-arvopareja ja on verrattavissa pythonin
dict-rakenteeseen.

%%%%%%%%%%%%%%%%%%%%%%%%%%%%%%%%%%%%%%%%%%%%%%%%%%%%%%%%%%%%%%%%%%%%%%%%%%%%%%%%%%%%%%%%%%%%%%%%%%

\section{Toistorakenteet}

\subsection{For-looppi}

\subsection{While-looppi}

\subsection{Foreach-looppi}

%%%%%%%%%%%%%%%%%%%%%%%%%%%%%%%%%%%%%%%%%%%%%%%%%%%%%%%%%%%%%%%%%%%%%%%%%%%%%%%%%%%%%%%%%%%%%%%%%%

\section{Javan ajamisen teoriaa}

\subsection{Tulkkaus vastaan kompilointi}

\subsection{Muistinhallinta ja garbage collector}

%%%%%%%%%%%%%%%%%%%%%%%%%%%%%%%%%%%%%%%%%%%%%%%%%%%%%%%%%%%%%%%%%%%%%%%%%%%%%%%%%%%%%%%%%%%%%%%%%%

\section{Lisää Javan konsepteja}

\subsection{Luokkamuuttujat, luokkafunktiot ja static}

\subsection{Tyyppimuunnokset (casting)}

\subsection{Vakiot ja final}

\subsection{Luetellut tyypit (enum)}

\subsection{Aritmeettiset- ja bittioperaatiot}

%%%%%%%%%%%%%%%%%%%%%%%%%%%%%%%%%%%%%%%%%%%%%%%%%%%%%%%%%%%%%%%%%%%%%%%%%%%%%%%%%%%%%%%%%%%%%%%%%%
% % % % % % % % % % % % % % % % % % % % % % % % % % % % % % % % % % % % % % % % % % % % % % % % %
%%%%%%%%%%%%%%%%%%%%%%%%%%%%%%%%%%%%%%%%%%%%%%%%%%%%%%%%%%%%%%%%%%%%%%%%%%%%%%%%%%%%%%%%%%%%%%%%%%

\clearpage
\printglossary[title=Sanasto, toctitle=Sanasto]
\printglossary[type=java, title=Javan avainsanat, toctitle=Javan avainsanat]

%%%%%%%%%%%%%%%%%%%%%%%%%%%%%%%%%%%%%%%%%%%%%%%%%%%%%%%%%%%%%%%%%%%%%%%%%%%%%%%%%%%%%%%%%%%%%%%%%%
% % % % % % % % % % % % % % % % % % % % % % % % % % % % % % % % % % % % % % % % % % % % % % % % %
%%%%%%%%%%%%%%%%%%%%%%%%%%%%%%%%%%%%%%%%%%%%%%%%%%%%%%%%%%%%%%%%%%%%%%%%%%%%%%%%%%%%%%%%%%%%%%%%%%

\end{document}
